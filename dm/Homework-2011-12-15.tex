\documentclass[a4paper]{scrartcl}
\usepackage[ngerman]{babel}
\usepackage[utf8]{inputenc}
\usepackage[T1]{fontenc}
\usepackage{lmodern}
\usepackage{amssymb}
\usepackage{amsmath}
\usepackage{enumerate}
\usepackage{scrpage2}\pagestyle{scrheadings}
\usepackage{tikz}

\newcommand{\titleinfo}{Hausaufgaben zum 15./16. Dezember 2011}

\title{\titleinfo}
\author{Elena Noll, Sven-Hendrik Haase, Arne Feil}
\date{\today}
\ihead{EN, SHH, AF}
\chead{\titleinfo}
\ohead{\today}
\setheadsepline{1pt}
\newcommand{\qed}{\quad \square}

\begin{document}
\maketitle

\begin{enumerate}
\item[\textbf{1.}]
\begin{enumerate}[a)]
\item Nicht isomorph, denn G hat keinen Knoten 3. Grades, der einen benachbarten Knoten 3. Grades hat. Bei G' ist dies aber der Fall.
\item Alle isomorph, da es sich um Varianten des Petersen-Graphen handelt.
\end{enumerate}
\item[\textbf{2.}]
\begin{enumerate}[a)]
\item \({10 \choose 2}\) = 45
\item \(\sum\limits_{i=1}^{i-3+1}i!\)
\item \(\sum\limits_{i=1}^{i-4+1}i!\)
\item 
\end{enumerate}
\item[\textbf{3.}]
\begin{enumerate}[a)]
\item Graph
\item Die Anzahl der Kanten von G lässt sich mit folgener Gleich berechnen \\
\[\binom{n}{2}+\frac{3}{2}n+n^2\]
Wobei \[\binom{n}{2}\] für die Anzahl der Kantn im vollständigen Graphen H2 steht,
\[\frac{3}{2}n\] für die Anzahl der Kanten von H1 steht und \[n^2\] für die Anzahl
der Verbindungskanten. \\
\\
Nun zeigen wir duch Umformung, dass unsere Gleichung der gegeben Gleichung
\[\frac{3}{2}n^2+n\] entspricht.
\[\binom{n}{2}+\frac{3}{2}n+n^2\]
\[\Rightarrow \frac{n^{\underline{2}}}{2!}+\frac{3}{2}n+n^2\]
\[\Rightarrow \frac{n(n-1)}{2 * 1} + \frac{3}{2}n + n^2\]
\[\Rightarrow \frac{n^2 - n}{2} + \frac{3}{2}n + n^2\]
\[\Rightarrow \frac{n^2 - n}{2} + \frac{3}{2}n + \frac{2n^2}{2}\]
\[\Rightarrow \frac{3n^2 - n}{2} + \frac{3n}{2}\]
\[\Rightarrow \frac{3}{2}n^2 + \frac{2n}{2}\]
\[\Rightarrow \frac{3}{2}n^2 + n\]
Da wir wissen, dass unsere aufgestellte Gleichung rictig ist, und wir gezeigt haben,
dass die beiden Gleichungen gleich sind, haben wir auch gezeigt, dass die gegebene 
Gleichung richtig ist.
\item Graph
\item G besitzt keine Eulersche Linie, da der Grad der Punkte von H1 immer ungerade ist.
Dies kann man dadurch begründen, dass der Grad der Punkte in H1 Anfangs immer 3 ist. 
Beim Verbinden von H1 und H2 zu G addiert man n zum bisherigen Grad 3 hinzu.
Da n immer eine gerade Zahl sein muss und eine ungerade Zahl plus eine gerade Zahl
immer eine ungerade Zahl ergibt, ist der Grad der Punkte von H1 in G auch immer ungerade.
Somit kann es sich um keine Eulersche Linie handeln, da die Voraussetzung für eine solche ist, dass jeder Punkt in einem Graphen einen geraden Grad hat.

\end{enumerate}
\item[\textbf{4.}]
\begin{enumerate}[a)]
\item
\end{enumerate}
%Ende aller Aufgaben
\end{enumerate}
\end{document}


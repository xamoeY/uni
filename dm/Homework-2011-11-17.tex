\documentclass[a4paper]{scrartcl}
\usepackage[ngerman]{babel}
\usepackage[utf8]{inputenc}
\usepackage[T1]{fontenc}
\usepackage{lmodern}
\usepackage{amssymb}
\usepackage{enumerate}
\usepackage{scrpage2}\pagestyle{scrheadings}

\newcommand{\titleinfo}{Hausaufgaben zum 17./18. November 2011}

\title{\titleinfo}
\author{Elena Noll, Sven-Hendrik Haase, Arne Feil}
\date{\today}
\ihead{EN, SHH, AF}
\chead{\titleinfo}
\ohead{\today}
\setheadsepline{1pt}
\newcommand{\qed}{\quad \square}

\begin{document}
\maketitle
\begin{enumerate}[1.]
\item
\begin{enumerate}[a)]
\item
%Es gibt für die Abbildung $g:X \rightarrow Y$ genau $7^5=16807$ Möglichkeiten $X$ 
%auf $Y$ abzubilden. Da für wir 5 mal aus einer 7-elementigen Menge ziehen und $g$ 
%und für ein mehrere Elemente aus $X$ das selbe Element aus $Y$ haben dürfen. 
%und es nicht unwichtig ist, ob ${1,2}$ gezogen wurden oder ${2,1}$.
%Wenn $g$ injektiv sein soll, ist es als würde man 5 mal aus einer 7-elementigen 
%Menge ziehen ohne wieder zurückzulegen. 
\item
${49 \choose 6}=13 983 816$
\item
$|M|=1000 \quad  997 \leq k \leq 1000 \\
{1000 \choose 997}+{1000 \choose 998}+{1000 \choose 999}+1=166 217 951$
\end{enumerate}
\item
\begin{enumerate}[a)]
\item
Der Koeffizient von $x^5y^{11}$ in $(x+y)^{16}$ lautet: ${16 \choose 5,11}= 
\frac{16!}{5!11!}=4368$ \\
Der Koeffizient von $x^3y^5z^2$ in $(x+y+z)^{10}$ lautet: ${10 \choose 3,5,2}= 
\frac{10!}{3!5!2!}=2520$
\item
$C_1 A P_1 P_2 U C_2 C_3 I N O$
$\frac{10!}{3!\cdot 2!}=302400$\\
$MANGOLAS_1S_2I
\frac{10!}{2!}=1 814 400$\\
$SE_1LTE_2R_1WAS_1S_2E_3R_2
\frac{12!}{2!\cdot 3!\cdot 2!}=19 958 400$
\item
Das Füllen der Kiste betrachten wir als Ziehen mit Zurücklegen ungeordnet von 
$k=6$ Flaschen aus $n=10$ Sorten.\\
${k+n-1 \choose k}={15 \choose 6}=5005$
\end{enumerate}
\item
Induktionanfang:\\
$n=3\\
{3 \choose 0}=1={4 \choose 4}$\\
Induktionannahme:\\
$\displaystyle \sum_{i=3}^{n}{i \choose i-3}={n+1 \choose 4}$\\
Induktionsschritt:\\
$\displaystyle \sum_{i=3}^{n+1}{i \choose i-3}=\sum_{i=3}^{n}{i \choose i-3}+{n+1 \choose n-2}$\\
Durch Verwenden der Induktionsannahme erhalten wir:\\
$\displaystyle {n+1 \choose 4}+{n+1 \choose n-2}={n+1 \choose 4}+{n+1 \choose 
n+1-(n-2)}={n+1 \choose 4}+{n+1 \choose 3}={n+2 \choose 4} \qed$
\item
\begin{enumerate}[a)]
\item

\item
\end{enumerate}
% Ende aller Aufgaben
\end{enumerate}
\end{document}

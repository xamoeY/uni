% Packages & Stuff
 
\documentclass[a4paper,11pt,parskip=half]{scrartcl}
\usepackage[ngerman]{babel}
\usepackage[utf8]{inputenc}
\usepackage[T1]{fontenc}
\usepackage[top=1.3in, bottom=1.2in, left=0.8in, right=0.8in]{geometry}
\usepackage{amsmath}
\usepackage{lmodern}
\usepackage{enumerate}
\usepackage{fancyhdr}
\usepackage{pgfplots}

% ------------------------------------------------------
 
% Commands
 
\newcommand{\authorinfo}{A. Struck, S. Haase, E. Böhmecke}
\newcommand{\titleinfo}{GSS-Übungsblatt 5 zum 18.06.2014}
\newcommand{\qed}{\ \square}
\newcommand{\todo}{\textcolor{red}{\textbf{TODO}}}
\newcommand{\opt}{\textcolor{blue}{\textbf{Optional}}}

% ------------------------------------------------------
 
% Title & Pages
 
\title{\titleinfo}
\author{\authorinfo}
 
\pagestyle{fancy}
\fancyhf{}
\fancyhead[L]{\authorinfo}
\fancyhead[R]{\titleinfo}
\fancyfoot[C]{\thepage}

\date{\today}

\begin{document}
\maketitle

\section*{Aufgabe 1}
\subsection*{1)}
Bei einem symmetrischen Kryptosystem besitzen beide 
Kommunikationspartner den gleichen Schluessel. D.h., es wird
der gleiche Schluessel zum ent- und verschluesseln verwendet.
\\
Im asymmetrischen Kryptosystem hingegen besitzen beide 
Kommunikationspartner jeweils ein eigenes Schluesselpaar aus
Public Key und Private Key. So kann jeder mit dem oeffentlich
zugaenglichen Public Key Dokumente verschluesseln, die nur dem
Besitzer des Private Key zugaenglich sind. Andersherum kann der
Besitzer des Private Keys Dokumente signieren, wobei jeder
mit Hilfe des Public Keys die Authentizitaet der Signatur
ueberpruefen kann.

\subsection*{2)}
\subsubsection*{a)}
Falls Alice Bob eine grosze Menge an Daten senden moechte,
waere es effiezenter, ein hybrides Kryptosystem einzusetzen.

\subsubsection*{b)}
Dabei wird ein symmetrischer Schluessel generiert, welche die
gesamten Nutzdaten verschluesselt. Der symmetrische Schluessel
selbst wird mit Alices asymmetrischen Schluessel verschluesselt
und ebenfalls uebertragen.

\subsubsection*{c)}
Den Nutzdaten der Nachricht geht der mit Bobs Public Key
verschluesselte symmetrische Schluessel voraus, mit dem die
anschliszenden Nutzdaten verschluesselt wurden.

\section*{Aufgabe 2}
\subsection*{2)}
Es scheint moeglich zu sein, sich die verguenstigungen auf
jedes Ticket ''aufzustempeln'', da diese nicht in eine größere
Checksumme aufgenommen werden. Es handelt sich um einfache
Zahlen, die an den eigentlich Code prependiert werden. \\
Angreifermodell:
\begin{itemize}
	\item Rolle: Benutzer
    \item Verbreitung: Systemfehler des Parkhauses (systemweit)
    \item Verhalten: aktiv (Druck eigener/Veränderung von vorhandenen Tickets)
    \item Rechenkapazität: Keine herkömmliche (vielleicht Drucker). Ansonsten so lange, wie der Angreifer braucht das Muster zu finden
\end{itemize}

\subsection*{3)}
Alle Daten muessen Teil einer groszen Checksumme
werden, welche dann mit einem dem System bekannten Schluessel
verschluesselt wird. Das Resultat davon wird als Barcode auf
die Karte gedruckt.

\section*{Aufgabe 3}
\subsection*{3.1.:}
Passiver Angriff: (abhören von Kommunikation von c) => Anmeldung (je nach \(E_k\)) dauerhaft oder einmalig durch Angreifer möglich, daten abfragen möglich
Aktiver Angriff: (Zugang, wie beim passivem Angriff), ändert Daten auf dem Server. 
	MiM-Angriff, tut so als ob der Angreifer der Server wäre. Kann durch Angriff auf den User \(k\) erlangen und somit durch \(c\) möglicherweise \(E_k\) reverse Engineeren.

\subsection{3.2.:}
Das reverse Engineering von \(E_k\) wird erschwert.
So lange r nicht serverseitig generiert UND sicher übertragen wird, ändert sich an Angriffen nichts

\subsection{3.3.:}
Durch Mitschneiden der gesamten Kommunikation ist das Abfangen der Kommunikation noch möglich


\section*{Aufgabe 5}
\subsection*{2)}
\begin{align}
\Phi(n) = (p-1)(q-1)\\
\Phi(n) = (281-1)(389-1)\\
\Phi(n) = 108640
\end{align}
Modulare inverse von 67 mod 108640 = d = 3243
\\\\
Der entschlüsselte Text lautet:
\\
Fuer die GSS-Klausur sind folgende Themen wichtig: Schutzziele, Angreifermodelle, Rainbow Tables, die (Un-)Sicherheit von Passwoertern und dazugehoerige Angriffe, Zugangs- und Zugriffskontrolle, Biometrische Verfahren, Timing-Attack und Power-Analysis, Grundlagen der KryptographieQ2 Authentifikationsprotokolle, das RSA-Verfahren und natuerlich alle anderen Inhalte, die wir in der Uebung und der Vorlesung behandelt haben :-)


\end{document}

% Packages & Stuff
 
\documentclass[a4paper,11pt,parskip=half]{scrartcl}
\usepackage[ngerman]{babel}
\usepackage[utf8]{inputenc}
\usepackage[T1]{fontenc}
\usepackage[top=1.3in, bottom=1.2in, left=0.8in, right=0.8in]{geometry}
\usepackage{amsmath}
\usepackage{lmodern}
\usepackage{enumerate}
\usepackage{fancyhdr}
\usepackage{pgfplots}

% ------------------------------------------------------
 
% Commands
 
\newcommand{\authorinfo}{A. Struck, S. Haase, E. Böhmecke}
\newcommand{\titleinfo}{GSS-Übungsblatt 6 zum 02.07.2014}
\newcommand{\qed}{\ \square}
\newcommand{\todo}{\textcolor{red}{\textbf{TODO}}}
\newcommand{\opt}{\textcolor{blue}{\textbf{Optional}}}

% ------------------------------------------------------
 
% Title & Pages
 
\title{\titleinfo}
\author{\authorinfo}
 
\pagestyle{fancy}
\fancyhf{}
\fancyhead[L]{\authorinfo}
\fancyhead[R]{\titleinfo}
\fancyfoot[C]{\thepage}

\date{\today}

\begin{document}
\maketitle

\section*{Aufgabe 1}
\subsection*{a)}
Bei der Größe handelt es sich um die Anzahl der virtuellen Adressen. Aus der Aufgabe geht hervor, dass \(2^{16}\) Adressen existieren und diese auf 16 Seiten verteilt werden müssen. Daraus ergibt sich: \(\frac{2^{16}B}{16} = 4096B\)

\subsection*{c)}
\(0x5fe8 \mod 16 = 8\) \\
\(0xfeee \mod 16 = 14\) \\
\(0xa470 \mod 16 = 0\) \\
\(0x0101 \mod 16 = 1\)

\subsection*{d)}
Kleine Seiten:
\begin{itemize}
	\item[i)] Da viele kleine Programme existieren und pro Progamm mindestens eine Seite genutzt wird, wären große Seitengrößen verschwendeter Platz. 
    \item[ii)] Sollte ein großes Programm etwas über der Seitengröße an Speicher benötigen, benötigt es ebenfalls eine Seite, die es kaum belegt
\end{itemize}
Große Seiten: 
\begin{itemize}
	\item[i)] Große Seitengrößen bedeuten insgesamt weniger Seiten, somit weniger Verwaltungsaufwand und weniger Zugriffszeit, beim wechseln einer Seite zur nächsten.
\end{itemize}


\subsection*{e)}
Laut Tanenbaum gibt es im Allgemeinen keine optimale Page Size,
er leitet jedoch für bekannte Prozessgrößen eine Formel her:
\(pagesize = \sqrt{2 \cdot avg\_processsize \cdot entrysize}\).\\
Diese ergibt sich aus der Ableitung der Speicherverbrauchsformel für einen Prozess: \\
\(Usage = \frac{avg\_processsize \cdot entrysize}{pagesize} + \frac{pagesize}{2}\), wobei \(\frac{pagesize}{2}\) den Verlust durch interne Fragmentierung und \(\frac{avg\_processsize \cdot entrysize}{pagesize}\) die Anzahl der Bytes in der Seitentabelle angiebt.

Damit haetten wir: \(\sqrt{2 \cdot 8B \cdot 4096000B} \approx 8096B\) (auf 2er Potenzen gerundet).

\section*{Aufgabe 2}
\subsection*{a)}
\subsubsection*{(a)}

\scalebox{0.8}{
\begin{tabular}{l||c|c|c|c|c|c|c|c|c|c|c|c|c|c|c|c|}
\hline
t 					& 1 & 2 & 3 & 4 & 5 & 6 & 7 & 8 & 9 & 10 & 11 & 12 & 13 & 14 & 15 \\
\hline
Seite   			& 1 & 2 & 3 & 4 & 5 & 6 & 1 & 3 & 1 &  6 &  3 &  5 &  4 &  2 &  1 \\
\hline
Fault   			& x & x & x & x & x & x & - & - & - &  - & -  & x  & x  & x  & -  \\
\hline
Seiten im Speicher  & 1 & 1,2 & 1,2,3 & 1,4,3 & 1,5,3 & 1,6,3 & 1,6,3 & 1,6,3 & 1,6,3 & 1,6,3 & 1,6,3 & 1,5,3 & 1,4,3 & 1,4,2 & 1,4,2 \\
\hline
\end{tabular}}

\subsubsection*{(b)}

\scalebox{0.8}{
\begin{tabular}{l||c|c|c|c|c|c|c|c|c|c|c|c|c|c|c|c|}
\hline
t 					& 1 & 2 & 3 & 4 & 5 & 6 & 7 & 8 & 9 & 10 & 11 & 12 & 13 & 14 & 15 \\
\hline
Seite   			& 1 & 2 & 3 & 4 & 5 & 6 & 1 & 3 & 1 &  6 &  3 &  5 &  4 &  2 &  1 \\
\hline
Fault   			& x & x & x & x & x & x & x & x & - &  - & -  &  x &  x &  x  & x  \\
\hline
Seiten im Speicher  & 1 & 1,2 & 1,2,3 & 2,3,4 & 3,4,5 & 4,5,6 & 5,6,1 & 6,1,3 & 6,3,1 &  3,1,6 & 1,6,3  & 6,3,5  &  3,5,4 & 5,4,2  & 4,2,1  \\
\hline
\end{tabular}}

\section*{Aufgabe 3}
\subsection*{a)}
\subsection*{b)}

\end{document}

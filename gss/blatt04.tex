% Packages & Stuff
 
\documentclass[a4paper,11pt,parskip=half]{scrartcl}
\usepackage[ngerman]{babel}
\usepackage[utf8]{inputenc}
\usepackage[T1]{fontenc}
\usepackage[top=1.3in, bottom=1.2in, left=0.8in, right=0.8in]{geometry}
\usepackage{amsmath}
\usepackage{lmodern}
\usepackage{enumerate}
\usepackage{fancyhdr}
\usepackage{pgfplots}
\usepackage{blockgraph}

% ------------------------------------------------------
 
% Commands
 
\newcommand{\authorinfo}{A. Struck, S. Haase, E. Böhmecke}
\newcommand{\titleinfo}{GSS-Übungsblatt 4 zum 28.05.2014}
\newcommand{\qed}{\ \square}
\newcommand{\todo}{\textcolor{red}{\textbf{TODO}}}
\newcommand{\opt}{\textcolor{blue}{\textbf{Optional}}}

% ------------------------------------------------------
 
% Title & Pages
 
\title{\titleinfo}
\author{\authorinfo}
 
\pagestyle{fancy}
\fancyhf{}
\fancyhead[L]{\authorinfo}
\fancyhead[R]{\titleinfo}
\fancyfoot[C]{\thepage}

\date{\today}

\begin{document}
\maketitle

\section*{Aufgabe 1}

\subsection*{a)}
\begin{blockgraph}{28}{1}{0.5}
    \bglabelxx{0}
    \bglabelxx{5}
    \bglabelxx{10}
    \bglabelxx{15}
    \bglabelxx{20}
    \bglabelxx{25}
    
    \bgemptysingleblock{0}
    \bgblock{1}{7}{$P_1$}
    \bgemptysingleblock{7}
    \bgblock{8}{9}{$P_3$}
    \bgemptysingleblock{9}
    \bgblock{10}{12}{$P_4$}
    \bgemptysingleblock{12}
    \bgblock{13}{18}{$P_2$}
    \bgemptysingleblock{18}
    \bgblock{19}{27}{$P_5$}
\end{blockgraph}

\subsection*{b)}
% blockgraph-Umgebung. Erzeugt den Blockgraphen. Parameter:
%   Breite (X-Achse)
%   Hoehe (Y-Achse)
%   Skalierung (Vergroeszerungsfaktor)
\begin{blockgraph}{35}{1}{0.4}
    % \bglabelxx erzeut Beschriftung der X-Achse an bestimmter Position
    \bglabelxx{0}
    \bglabelxx{5}
    \bglabelxx{10}
    \bglabelxx{15}
    \bglabelxx{20}
    \bglabelxx{25}
    \bglabelxx{30}
    \bglabelxx{35}
    
    % \bgblock erzeugt Block innerhalb des Graphen. Parameter:
    %    Y-Position (z.B. CPU), optional
    %    Beginn auf der X-Achse
    %    Ende auf der X-Achse
    %    Beschriftung
    \bgemptysingleblock{0}
    \bgblock{1}{3}{$P_1$}
    \bgblock{3}{5}{$P_1$}
    \bgemptysingleblock{5}
    \bgblock{6}{8}{$P_2$}
    \bgemptysingleblock{8}
    \bgblock{9}{10}{$P_3$}
    \bgemptysingleblock{10}
    \bgblock{11}{13}{$P_1$}
    \bgemptysingleblock{13}
    \bgblock{14}{16}{$P_4$}
    \bgemptysingleblock{16}
    \bgblock{17}{19}{$P_2$}
    \bgemptysingleblock{19}
    \bgblock{20}{22}{$P_5$}
    \bgemptysingleblock{22}
    \bgblock{23}{24}{$P_2$}
    \bgemptysingleblock{24}
    \bgblock{25}{27}{$P_5$}
    \bgblock{27}{29}{$P_5$}
    \bgblock{29}{31}{$P_5$}

\end{blockgraph}


\section*{Aufgabe 2}

\subsection*{a)}
Annahme: Wenn ein idealer Scheduler alle Deadlines in unbeschränkter Zeit einhalten soll, muss ein Zeitintervall \((I)\) existieren, für das gilt: \\
a) Es ist (ganzzahlig) durch die Periodendauern \((Pd(A_i))\) teilbar. \\
b) Die Summe der Bedienzeiten \((B(A_i))\) im Zeitintervall ist kleiner, als die Länge des Intervalles. \\
\\
Da \(B(A_i)\) durch die Anzahl der Perioden im Intervall \((PI(A_i))\) in Abhängigkeit von \(|I|\) steht, ist das kleinste gemeinsame Vielfache der Periodenlängen praktikabel. Daher:\\ 
\(|I| = kgV(4,7,3) = 84 \) \\ \newpage
Für \(PI(A_i)\) gilt: \\
\(PI(A_i) = |I|: Pd(A_i)\) \\
\begin{center}
\(
\begin{array}{c | c c}
	Task & PI(A_i) & B(A_i) \\ \hline
	A_1  & 21 & 21 \\
	A_2  & 12 & 36 \\ 
	A_3  & 24 & 24
\end{array}\quad \Rightarrow \quad
 \sum\limits_{i=1}^3B(A_i) = 85 > 84 = |I|
\)
\end{center}
Somit ist gezeigt, dass es selbst für einen perfekten Scheduler unmöglich ist, alle Deadlines bei unendlicher Laufzeit einzuhalten.

\subsection*{b) ii)}
\begin{blockgraph}{25}{1}{0.5}
    \bglabelxx{0}
    \bglabelxx{5}
    \bglabelxx{10}
    \bglabelxx{15}
    \bglabelxx{20}
    \bglabelxx{25}
    
    \bgblock{0}{3}{$B_1$}
    \bgblock{3}{4}{$B_2$}
    \bgblock{4}{6}{$B_3$}
    \bgblock{6}{7}{$B_4$}
    \bgblock{7}{10}{$B_1$}
    \bgblock{10}{11}{$B_4$}
    \bgblock{12}{14}{$B_3$}
    \bgblock{14}{15}{$B_2$}
    \bgblock{15}{18}{$B_1$}
    \bgblock{18}{19}{$B_4$}
    \bgblock{21}{22}{$B_3$}
    \bgblock{22}{25}{$B_1$}
\end{blockgraph}

\subsection*{c)}

Interpretation: Hoehere Prioritaet ist besser.

% blockgraph-Umgebung. Erzeugt den Blockgraphen. Parameter:
%   Breite (X-Achse)
%   Hoehe (Y-Achse)
%   Skalierung (Vergroeszerungsfaktor)
\begin{blockgraph}{20}{4}{0.7}
        % \bglabelxx erzeut Beschriftung der X-Achse an bestimmter Position, wobei Position und Beschriftung identisch sind
        \bglabelxx{0}
        \bglabelxx{5}
        \bglabelxx{10}
        \bglabelxx{15}
        \bglabelxx{20}

        % \bglabely erzeut Beschriftung der X-Achse an bestimmter Position
        \bglabely{0}{CPU 0}
        \bglabely{1}{CPU 1}
        \bglabely{2}{CPU 2}
        \bglabely{3}{CPU 3}

        % \bgblock erzeugt Block innerhalb des Graphen. Parameter:
        %    Y-Position (z.B. CPU), optional
        %    Beginn auf der X-Achse
        %    Ende auf der X-Achse
        %    Beschriftung
        %\bgemptysingleblock[2]{6}

        \bgblock[0]{0}{4}{$P_1$}
        \bgblock[1]{2}{8}{$P_2$}
        \bgblock[2]{2}{7}{$P_3$}
        \bgblock[3]{3}{11}{$P_5$}
        \bgblock[0]{4}{5}{$P_4$}
        \bgblock[0]{5}{10}{$P_6$}
        \bgblock[2]{7}{14}{$P_4$}
        \bgblock[1]{9}{13}{$P_7$}
\end{blockgraph}

\section*{Aufgabe 3}

\subsection*{a)}


\end{document}

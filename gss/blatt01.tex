% Packages & Stuff
 
\documentclass[a4paper,11pt]{scrartcl}
\usepackage[ngerman]{babel}
\usepackage[utf8]{inputenc}
\usepackage[T1]{fontenc}
\usepackage[top=1.3in, bottom=1.2in, left=0.9in, right=0.9in]{geometry}
\usepackage{lmodern}
\usepackage{amssymb}
\usepackage{amsmath}
\usepackage{enumerate}
\usepackage{fancyhdr}
\usepackage{pgfplots}
\usepackage{multicol}
 
% ------------------------------------------------------
 
% Commands
 
\newcommand{\authorinfo}{A. Struck, A. Dammer, S. Haase, E. Böhmecke}
\newcommand{\titleinfo}{GSS-Übungsblatt 1 zum 16.04.2014}
\newcommand{\qed}{\ \square}
\newcommand{\todo}{\textcolor{red}{\textbf{TODO}}}
\newcommand{\opt}{\textcolor{blue}{\textbf{Optional}}}
 
% ------------------------------------------------------
 
% Title & Pages
 
\title{\titleinfo}
\author{\authorinfo}
 
\pagestyle{fancy}
\fancyhf{}
\fancyhead[L]{\authorinfo}
\fancyhead[R]{\titleinfo}
\fancyfoot[C]{\thepage}
 
\begin{document}
\maketitle

\section*{Aufgabe 1}
\subsection*{1.}
Ein nicht-verteiltes System benötigt kein Netzwerk, um effizient Rechnen zu
können. Damit ist es nicht von außen angreifbar. Ein verteiltes System teilt sich
mit vielen anderen Rechnern privaten Speicher. Damit können gleichzeitig viele
Systeme angegriffen werden, falls eines der Systeme kompromitiert wurde.

\subsection*{2.}
\begin{itemize}
	\item wenig Zeit und Rücksicht für eigene IT-Infrastruktur
    \item Inkomptetenz
    \item Profitstreben auf kurze Zeiträume
\end{itemize}

\subsection*{3.}
\subsubsection*{a)}
\opt

\subsubsection*{b)}
\opt
         
\section*{Aufgabe 2}
\subsection*{1.}
\begin{itemize}
\item{a)} 
Anonymität: Nutzer können Ressourcen und Dienste benutzen, ohne ihre 
Indetität zu offenbaren. Selbst der Kommunikationspartner erfährt nicht die Identität.\\
Pseudonymität: Nutzer können Ressourcen und Dienste benutzen, ohne echte Identität zu veröffentlichen, sondern ein Kennzeichen (Pseudonym) wird verwendet. (Pseudonym ist mit der Identität verknüpft) \\
Unbeobachbarkeit: Nutzer können Ressourcen und Dienste benutzen, ohne dass andere dies beobachten können. Dritte können weder das Sensen noch den Erhalt von Nachrichten beobachten.\\
\\
Manche Anwendung erfordert trotz anonymer und unbeobachtbarer Kommunikation auch
die Zurechenbarkeit von Aktionen zu ihrem Akteur. Pseudonymität
gestattet die Verknüpfung von Anonymität und Zurechenbarkeit.
\\
\item{b)} Vertraulichkeit: Geheimhaltung von Daten während der Übertragung. \\
Verdecktheit: Versteckte Übertragung von vertraulichen Daten. \\
Die Vertraulichkeit bedeutet, dass nur Kommunikationspartnern den Inhalt der Nachricht erkennen können. Und im Vergleich zu Vertraulichkeit bedeutet  die Verdecktheit, dass Nachrictaustausch "beschutzt" ist - nur Kommunikationspartnern wissen, dass Nachricht existiert.


\end{itemize}

\subsection*{2.}
\opt
\subsection*{3.}
\opt
 
\section*{Aufgabe 3}
\subsection*{1.}
Ein Angreifermodell ist ein Modell, welches den stärksten Angreifer auf ein System (mit einem in dem 
Modell spezifizierten Angriffsvektor), den der Systemschutz noch gerade abwehren kann. Ein solches Modell 
wird aufgestellt, um aufzuzeigen, wie gut bzw. schlecht ein System ungefähr geschützt ist. \\
Es werden die folgenden Kriterien berücksichtigt:
\begin{itemize}
  \item \textbf{Rollen:} 
  Unterteilt in In- und Outsider, enthält das Rollen-Kriterium die Position des Angreifers in Relation zum
  System.
  \item \textbf{Verbreitung:} 
  Beschreibt die Verbreitung des Angriffs, beispielsweise lokal oder netzweit.
  \item \textbf{Verhalten:} 
  Unterteilt in aktiv und passiv bzw. beobachtend und verändernd, beschreibt das Verhaltens-Kriterium, welche Eingriffe vorgenommen werden.
  \item \textbf{Rechenkapazitäten:} 
  Beschreibt die ungefähre Rechenkapazität des Angreifenden (Ausprägungen sind beispielsweise beschränkt 
  und unbeschränkt)
\end{itemize}       

\subsection*{2.}
\opt

\section*{Aufgabe 4}
\subsection*{1.}
Kennwortspeicherung:
Bei Kennwortspeicherung wird das Passwort des Benutzers in eine Hash-Funktion umgewandelt und zusammen mit dem Benutzernamen abgespeichert.
Kennwortprüfung:
Bei der Kennwortprüfung wird überprüft ob der eingegebene Benutzername mit dem abgespeicherten Benutzernamen übereinstimmt. Zusätzlich wird das eingegebene Passwort in eine Hash-Funktion umgewandelt und mit der abgespeicherten Hash-Funtkion verglichen. Wenn die Benutzername – Hash-Funktion Kombination mit der abgespeicherten übereinstimmt, war die Anmeldung erfolgreich.

User: leroy
Passwort: hund
\subsection*{2.}
\opt
\subsection*{3.}
\opt
\subsection*{4.}
\opt
\subsection*{5.}
\opt
\end{document}

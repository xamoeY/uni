\documentclass[
  twoside,
  11pt, a4paper,
  footinclude=true,
  headinclude=true,
  cleardoublepage=empty
]{scrreprt}

\usepackage{lipsum}
\usepackage[ngerman,english]{babel}
\usepackage{amsmath}
\usepackage{amsthm}
\usepackage{graphicx}
\usepackage{listings}
\usepackage{lmodern}
\usepackage{setspace}
\usepackage[T1]{fontenc}
\usepackage[utf8]{inputenc}

\begin{document}

\begin{titlepage}
    \begin{center}
        \Huge
        \textbf{Analysis Of the Feasibility Of Pathtracing In Interactive Media}
        
        \vspace{0.5cm}

        \LARGE
        \textit{Or:} How Much Noise Is Too Much?
        
        \vspace{1.5cm}
        
        \textbf{Sven-Hendrik Haase}
        
        \vfill
        
        A thesis presented for the degree of\\
        \emph{Bachelor of Computer Science}
        
        \vspace{0.8cm}
        
        %\includegraphics[width=0.4\textwidth]{logo_uhh.jpg}
        \includegraphics[width=0.4\textwidth]{frontbackmatter/uni-siegel.png}
        
        \Large
        Department of Informatics\\
        University of Hamburg\\
        Germany\\
        2015-20-08

        \vspace{0.8cm}

        \large
        Primary Supervisor: Prof. Dr. rer. nat. Leonie \textsc{Dreschler-Fischer}\\
        Secondary Supervisor: \textsc{TBD}
        
    \end{center}
\end{titlepage}


\chapter*{Abstract}
\onehalfspace
As part of the quest for ever-improving game graphics, researchers, graphics hardware developers
and video game developers alike have been coming up with more and more convoluted and technically
challenging ways of improving the graphics in interactive media such as games in order to give
users a deeper sense of immersion or to provide special effects artists with faster feedback.

While rendering techniques are currently shifting from the traditional fixed pipeline approach
towards the new, fully programmable approach that lets developers implement deferred renderers that
can more closely mimic reality by using multiple combined shading and lighting algorithms, the
fundamental concept of rasterization-based rendering has remained the same.

The real world photon-collecting approach that actual cameras use has so far not been adopted for
interactive media by the industry in any capacity because the computational cost has historically
been prohibitively expensive.
This paper presents the results of my research about the feasibility of using a physically-based
technique called \textbf{path tracing} in lieu of or in corporation with classical techniques in interactive media. I
present the idea, algorithm and complexity behind path tracing and extrapolate feasibility
according to the current state of technology and trends. As part of the research, I have implemented a path tracing 3D engine in C++.

In general, I have found path tracing to be a viable rendering solution on commodity hardware in
about 6 years.
\singlespace

\chapter*{Acknowledgements}
\doublespacing
I would like to express my sincere gratitude to the teachers throughout school and university for
the knowledge they've passed on.

I thank my friends for the laughs we've shared and for the experiences we've made and had together.

Furthermore, none of this would have been possible without the incredible efforts and love of my
parents who have supported me throughout the years and enabled me to live a carefree life until I
was ready to fend for myself.

Last, but certainly not least, I would like to declare my gratefulness to Alisa, whose endless love
has given my life a new meaning.

\singlespace

\clearpage
\vspace*{\fill}
\thispagestyle{empty} % suppress showing of page number
\begin{quotation}
    \em
    We all make choices in life, but in the end, our choices make us.

    \medskip
    \raggedleft
    Andrew Ryan
\end{quotation}
\vspace*{\fill}

\tableofcontents

\chapter{Introduction}
\lipsum[1]

\section{Motivation}
\lipsum[1]

\chapter{Path Tracing Explained}
\lipsum[1]

\section{Theoretical Basis}
\lipsum[1]

\section{Path Tracing in Comparison to Other Techniques}
\lipsum[1]

\section{History of Path Tracing}
\lipsum[1]

\section{Current State of Technology}
\lipsum[1]

\chapter{Research}
\lipsum[1]

\section{Results}
\lipsum[1]

\section{Evaluation}
\lipsum[1]

\chapter{Conclusion}
\lipsum[1]

\section{Outlook}
\lipsum[1]

\listoffigures
 
\listoftables

\begin{thebibliography}{9}
    \bibitem{lamport94}
        Leslie Lamport,
        \emph{\LaTeX: a document preparation system},
        Addison Wesley, Massachusetts,
        2nd edition,
        1994.
\end{thebibliography}
    
\end{document}

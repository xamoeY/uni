\documentclass[
  twoside,
  11pt, a4paper,
  footinclude=true,
  headinclude=true,
  cleardoublepage=empty
]{scrreprt}

\usepackage{lipsum}
\usepackage[ngerman,english]{babel}
\usepackage{amsmath}
\usepackage{amsthm}
\usepackage{graphicx}
\usepackage{listings}
\usepackage{lmodern}
\usepackage{setspace}
\usepackage{url}
\usepackage[T1]{fontenc}
\usepackage[utf8]{inputenc}

\begin{document}

\begin{titlepage}
    \begin{center}
        \Huge
        \textbf{Analysis Of the Feasibility Of Pathtracing In Interactive Media}
        
        \vspace{0.5cm}

        \LARGE
        \textit{Or:} How Much Noise Is Too Much?
        
        \vspace{1.5cm}
        
        \textbf{Sven-Hendrik Haase}
        
        \vfill
        
        A thesis presented for the degree of\\
        \emph{Bachelor of Computer Science}
        
        \vspace{0.8cm}
        
        %\includegraphics[width=0.4\textwidth]{logo_uhh.jpg}
        \includegraphics[width=0.4\textwidth]{frontbackmatter/uni-siegel.png}
        
        \Large
        Department of Informatics\\
        University of Hamburg\\
        Germany\\
        2015-20-08

        \vspace{0.8cm}

        \large
        Primary Supervisor: Prof. Dr. rer. nat. Leonie \textsc{Dreschler-Fischer}\\
        Secondary Supervisor: \textsc{TBD}
        
    \end{center}
\end{titlepage}


\chapter*{Abstract}
\onehalfspace
This study aims to investigate the viability of a physically-based technique called
\textbf{path tracing} in lieu of or in corporation with classical techniques in interactive media such as video
games and visual effects tools.

Real time path tracing has been prohibitively expensive in regards to computational
complexity. However, modern GPUs and even CPUs have finally gotten fast enough for real time path
tracing to become a viable alternative to traditional real time approaches to rendering.  Based on
that assumption, this thesis presents the idea, algorithm and complexity behind path tracing in the
first part and extrapolates feasibility and suitability of real time path tracing on consumer
hardware according to the current state of technology and trends in the second part.

As part of the research, the author has implemented a path tracing 3D engine in modern C++ in order
to empirically test the assumptions made in this thesis. The study found path tracing
to be a viable rendering technique for average commodity hardware in approximately 4 years.
\singlespace

\chapter*{Acknowledgments}
\doublespacing
I would like to express my sincere gratitude to the teachers throughout school and university for
the knowledge they've passed on.

I thank my friends for the laughs, horrible mistakes and awesome successes we shared with
one another.

Furthermore, none of this would have been possible without the incredible efforts and love of my
parents who have supported me throughout the years and enabled me to live a carefree life until I
was ready to fend for myself.

Lastly, but certainly not least, I would like to declare my gratefulness to Alisa, whose endless love
has given my life a new meaning.

\singlespace

\clearpage
\vspace*{\fill}
\thispagestyle{empty} % suppress showing of page number
\begin{quotation}
    \em
    We all make choices in life, but in the end, our choices make us.

    \medskip
    \raggedleft
    Andrew Ryan
\end{quotation}
\vspace*{\fill}

\tableofcontents

\chapter{Introduction}
As part of the quest for ever-improving game graphics, researchers, graphics hardware developers
and video game developers alike have been coming up with more and more convoluted and technically
challenging ways of improving the graphics in interactive media such as games and visualizations in order to give
users a deeper sense of immersion or to provide special effects artists with faster feedback.

While rendering techniques are currently shifting from the traditional fixed pipeline approach
towards the new, fully programmable approach that lets developers implement deferred renderers that
can more closely mimic reality by using multiple combined shading and lighting algorithms and
rendering the scene multiple times for different buffers, the
fundamental concept of rasterization-based rendering has largely remained the same.

The real world photon-collecting approach that actual cameras use has so far not been adopted for
interactive media by the industry in any capacity because the computational cost has historically
been prohibitively expensive. It is, however, used extensively (and has been in use since decades)
for offline, non-interactive rendering of computer-generated movies and visualizations of scientific simulations.

This study assumes that the next logical step for the industry will be to adopt
this method for real time media as well. For the purpose of this thesis, a renderer is considered
\textit{real time} when it manages to render a frame within \(16.67ms\) since that equals 60 frames per second
which is the current de-facto standard for most available computer screens.

\section{Motivation}
Real time path tracing (and physically based rendering in general) offers many
benefits over traditional real time rendering methods such as better visuals
and simpler implementation but also allows for completely new types of graphics
such as realistic caustics \cite{wiki:caustics} and even light dispersion
\cite{wiki:dispersion} (using a prism, for instance) since path tracers might simulate wavelenghts
instead of plain RGB colors. Modern video games tend
to rely on a growing number of tricks to keep them visually appealing as the
consumer grows more demanding. They're called \textit{tricks} in this study
because they merely trick the beholder into seeing something that appears to be
physically accurate when it is, in fact, not the result of a physically-based
calculation and as such this study aims to keep tricks and emergent phenomena
separated by language. Some notable tricks include screen-space ambient occlusion (SSAO)
\cite{wiki:ssao}, motion blur \cite{wiki:motion-blur}, lens flare
\cite{wiki:lens-flare}, chromatic aberration \cite{wiki:chromatic-aberration},
depth of field \cite{wiki:depth-of-field} and light mapping \cite{wiki:lightmap}.

\chapter{Path Tracing Explained}
\lipsum[1]

\section{Theoretical Basis}
\lipsum[1]

\section{Path Tracing in Comparison to Other Techniques}
\lipsum[1]

\section{History of Path Tracing}
omg

\section{Current State of Technology}
lol
\lipsum[1]

\chapter{Research}
\lipsum[1]

\section{Results}
\lipsum[1]

\section{Evaluation}
\lipsum[1]

\chapter{Conclusion}
\lipsum[1]

\section{Outlook}
\lipsum[1]

\listoffigures
 
\listoftables

\bibliographystyle{unsrt}
\bibliography{main}
    
\end{document}

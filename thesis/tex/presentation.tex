\documentclass{beamer}

\usepackage{lipsum}
\usepackage[utf8]{inputenc}
\usepackage[ngerman,english]{babel}
\usepackage{amsmath}
\usepackage{amsthm}
\usepackage{graphicx}
\usepackage{caption}
\usepackage{lmodern}
\usepackage{float}
\usepackage{sidecap}
\usepackage{pgfplots}
\usepackage{pgfplotstable}
\usepackage{tabularcalc}
\usepackage{todonotes}
\usepackage{hyperref}
\usepackage{minted}
% \usepackage{siunitx}
% \usepackage{acronym}
% \usepackage{subfig}
% \usepackage{tabularx}
% \usepackage{setspace}
\usepackage[customcolors]{hf-tikz}
% \usepackage{url}
% \usepackage{csquotes}
% \usepackage{booktabs}
% \usepackage[T1]{fontenc}
% \usepackage[alldates=long]{biblatex}
\graphicspath{ {images/} }
\pgfplotsset{compat=1.12}

\title[Path Tracing]
{Feasibility Study of Real Time Path Tracing}
\subtitle{\textit{Or:} How Much Noise Is Too Much?}
\author[Haase]{Sven-Hendrik Haase \\ \footnotesize Matriculation number: 6341873}
\institute[University of Hamburg]{
    Department of Computer Science\\
    University of Hamburg
}
\subject{Computer Science}

\usetheme{Szeged}
\usecolortheme{seagull}

\begin{document}
\frame{\titlepage}
\begin{frame}
    \frametitle{Table of Contents}
    \tableofcontents
\end{frame}

\section{Introduction}

\subsection{Motivation}
\begin{frame}
    \frametitle{Motivation}
    \begin{itemize}
        \item Current generation graphics made up of many complex tricks
        \pause
        \item Path tracing is simple in comparison
        \pause
        \item Superior graphics quality
        \pause
        \item Allows for simulation of caustics, global illumination, light dispersion, etc.
        \pause
        \item Real time path tracing appears to be in reach
    \end{itemize}
\end{frame}

\subsection{Leading Question and Goals}
\begin{frame}
    \frametitle{Leading Question and Goals}
    \begin{itemize}
        \item Find out when real time path tracing will be viable
        \item Theoretical indicators (GPU peak FLOPS)
        \item Practical indicators (benchmarks)
        \item Prefer performance to quality
    \end{itemize}
\end{frame}

\section{Real Time Path Tracing Explained}

\subsection{Physically Based Approach}
\begin{frame}
    \frametitle{Physically Based Approach}
    \begin{block}{Forward path tracing}
        Light source (\(\Rightarrow\) scene interactions) \(\Rightarrow\) observer
    \end{block}
    \begin{block}{Backward path tracing}
        Observer (\(\Rightarrow\) scene interactions) \(\Rightarrow\) light source
    \end{block}
\end{frame}

\subsection{Theoretical Basis}
\begin{frame}
    \frametitle{Theoretical Basis}
    \begin{block}{James Kajiya's rendering equation}
        \scriptsize
        \(
            L_{\text{o}}(\mathbf x,\, \omega_{\text{o}},\, \lambda,\, t) \,=
            \, L_{\text{e}}(\mathbf x,\, \omega_{\text{o}},\, \lambda,\, t) \ +
            \, \int_\Omega f_{\text{r}}(\mathbf x,\, \omega_{\text{i}},\ \omega_{\text{o}},\, \lambda,\, t)
            \, L_{\text{i}}(\mathbf x,\, \omega_{\text{i}},\, \lambda,\, t)\,
            (\omega_{\text{i}}\,\cdot\,\mathbf n)\, \operatorname d \omega_{\text{i}}
        \)
    \end{block}
    \begin{block}{Simplified rendering equation}
        \(
            L_{\text{o}}(\mathbf x,\, \omega_{\text{o}}) \,=
            \, L_{\text{e}}(\mathbf x,\, \omega_{\text{o}}) \ +
            \, \int_\Omega f_{\text{r}}(\mathbf x,\, \omega_{\text{i}},\ \omega_{\text{o}})
            \, L_{\text{i}}(\mathbf x,\, \omega_{\text{i}})\,
            (\omega_{\text{i}}\,\cdot\,\mathbf n)\, \operatorname d \omega_{\text{i}}
        \)
    \end{block}

    \begin{itemize}
        \item Path tracing is a numerical approximate solution to the rendering equation
    \end{itemize}
\end{frame}

\begin{frame}
    \frametitle{Rendering equation breakdown}
\[
    \tikzmarkin[set fill color=red!50!brown!30,set border color=red!40!black]{outgoing}
    L_{\text{o}}(\mathbf x,\, \omega_{\text{o}})
    \tikzmarkend{outgoing}
    \, = \,
    \tikzmarkin[set fill color=cyan!70!lime!30,set border color=cyan!40!black]{emitted}
    L_{\text{e}}(\mathbf x,\, \omega_{\text{o}})
    \tikzmarkend{emitted}
    \ + \,
    \tikzmarkin[set fill color=yellow!50!lime!30,set border color=yellow!40!black]{integral}(0.1,-0.4)(-0.1,0.6)
    \int_\Omega
    \tikzmarkin[set fill color=green!50!lime!30,set border color=green!40!black]{brdf}(0.0,-0.2)(-0.0,0.4)
    f_{\text{r}}(\mathbf x,\, \omega_{\text{i}},\ \omega_{\text{o}})
    \tikzmarkend{brdf}
    \,
    \tikzmarkin[set fill color=blue!50!lime!30,set border color=blue!40!black]{incoming}(0.0,-0.2)(-0.0,0.4)
    L_{\text{i}}(\mathbf x,\, \omega_{\text{i}})
    \tikzmarkend{incoming}
    \,
    \tikzmarkin[set fill color=magenta!100!lime!30,set border color=pink!40!black]{attenuation}(0.0,-0.2)(-0.0,0.4)
    (\omega_{\text{i}}\,\cdot\,\mathbf n)
    \tikzmarkend{attenuation}
    \,
    \operatorname d \omega_{\text{i}}
    \tikzmarkend{integral}
\]

\(
    \tikzmarkin[set fill color=red!50!brown!30,set border color=red!40!black]{outgoing'}(0.1,-0.15)(-0.1,0.35)
    L_{\text{o}}(\mathbf x,\, \omega_{\text{o}})
    \tikzmarkend{outgoing'}
\)\, is the \textbf{outgoing light} with \(\mathbf x\) being a point on a surface from which the light is
reflected from into direction \(\omega_{\text{o}}\).

\(
    \tikzmarkin[set fill color=cyan!70!lime!30,set border color=cyan!40!black]{emitted'}(0.1,-0.15)(-0.1,0.35)
    L_{\text{e}}(\mathbf x,\, \omega_{\text{o}})
    \tikzmarkend{emitted'}
\)\, is the \textbf{emitted light} from point \(\mathbf x\).
    \note{Usually surfaces don't emit light themselves unless they are area lights.}

\smallskip
\(
    \tikzmarkin[set fill color=yellow!50!lime!30,set border color=yellow!40!black]{integral'}(0.1,-0.2)(-0.1,0.35)
    \int_\Omega \, \ldots \, \operatorname d \omega_{\text{i}}
    \tikzmarkend{integral'}
\)\, is the integral over \(\Omega\) which is the hemisphere at \(\mathbf x\) (centered around \(\mathbf n\)). All possible values for \(\omega_{\text{i}}\) are therefore
contained in \(\Omega\).

\(
    \tikzmarkin[set fill color=green!50!lime!30,set border color=green!40!black]{brdf'}(0.1,-0.15)(-0.1,0.35)
        f_{\text{r}}(\mathbf x,\, \omega_{\text{i}},\ \omega_{\text{o}})
    \tikzmarkend{brdf'}
    \)\, is the \textbf{BRDF} which determines how much light is reflected from
\(\omega_{\text{i}}\) to \(\omega_{\text{o}}\) at \(\mathbf x\).

\(
    \tikzmarkin[set fill color=blue!50!lime!30,set border color=blue!40!black]{incoming'}(0.1,-0.15)(-0.1,0.35)
        L_{\text{i}}(\mathbf x,\, \omega_{\text{i}})
    \tikzmarkend{incoming'}
\)\, is the \textbf{incoming light} at \(\mathbf x\) from \(\omega_{\text{o}}\).
    \note{It is not necessarily \emph{direct light}. The rendering equation also considers
    \emph{indirect light} which is light that has already been reflected.}

\smallskip
\(
    \tikzmarkin[set fill color=magenta!100!lime!30,set border color=pink!40!black]{attenuation'}(0.1,-0.15)(-0.1,0.35)
        (\omega_{\text{i}}\,\cdot\,\mathbf n)
    \tikzmarkend{attenuation'}
\)\, is the \textbf{normal attenuation} at \(\mathbf x\).
    \note{The incoming light \(\omega_{\text{i}}\) is
    weakened depending on the cosine of the angle between \(\omega_{\text{i}}\) and the surface normal
    \(\mathbf n\).}
\end{frame}

\subsection{Algorithm}
\begin{frame}[fragile]
    \frametitle{Algorithm part 1}
    \scriptsize
    \begin{minted}[linenos, autogobble]{python}
        max_depth = 5
        scene = [triangle_1, ..., triangle_n]  # Many triangles defined here

        def trace_ray(ray, depth):
            if depth >= max_depth:
                # Return black since we haven't hit anything but we're
                # at our limit for bounces
                return RGB(0, 0, 0) 

            intersection = None
            for triangle in scene:
                intersection = check_intersection(ray, triangle)
                if intersection:  # Break at first intersection
                    break

            if not intersection:
                # If we haven't hit anything, we can't bounce again so
                # we return black
                return RGB(0, 0, 0)
    \end{minted}
\end{frame}

\begin{frame}[fragile]
    \frametitle{Algorithm part 2}
    \scriptsize
    \begin{minted}[linenos, autogobble, firstnumber=20]{python}
            material = intersection.material;
            emittance = material.emittance

            # Shoot a ray into random direction and recurse
            next_ray = Ray()
            next_ray.origin = intersection.position
            next_ray.direction = random_vector_on_hemisphere(intersection.normal)

            # BRDF for diffuse materials
            reflectance_theta = dot(next_ray.direction, intersection.normal)
            brdf = 2 * material.reflectance * reflectance_theta
            reflected = trace_ray(next_ray, depth + 1)

            return emittance + (brdf * reflected)

        for sample in range(samples):
            for pixel in pixels:
                trace_path(ray_from_pixel(pixel), 0)
    \end{minted}
\end{frame}

\subsection{History of Path Tracing}
\begin{frame}
    \frametitle{Motivation}
    \lipsum[2]
\end{frame}

\subsection{Current State of Technology}
\begin{frame}
    \frametitle{Motivation}
    \lipsum[2]
\end{frame}

\section{Research}
\subsection{Implementation Design Overview}
\begin{frame}
    \frametitle{Design Overview}
    \begin{enumerate}
        \item herp
        \item derp
    \end{enumerate}
\end{frame}

\subsection{Results}
\begin{frame}
    \frametitle{Results}
    \begin{enumerate}
        \item herp
        \item derp
    \end{enumerate}
\end{frame}

\subsection{Evaluation}
\begin{frame}
    \frametitle{Evaluation}
    \begin{enumerate}
        \item herp
        \item derp
    \end{enumerate}
\end{frame}

\section{Conclusion and Outlook}
\subsection{Conclusion}
\begin{frame}
    \frametitle{Conclusion}
    \begin{itemize}
        \item Recall the goal: When will real time path tracing be viable on commodity hardware?
        \item Real time means 60 FPS
        \pause
        \item 
    \end{itemize}
\end{frame}

\subsection{Outlook}
\begin{frame}
    \frametitle{Outlook}
    Many venues for improvement:
    \pause
    \begin{itemize}
        \item Overall: Faster hardware
        \pause
        \item Software: Memory access optimizations
        \pause
        \item Intersection: Acceleration Data Structure (BVH, kd-tree)
        \pause
        \item Convergence: Bidirectional path tracing + Multiple Importance Sampling
        \pause
        \item Image filtering: Bilateral filter
        \pause
        \item Research: Who knows?
    \end{itemize}
    \pause
    \(\Longrightarrow\) Real time path tracing possible within 4 years
\end{frame}
\end{document}

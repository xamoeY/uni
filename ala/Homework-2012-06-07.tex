\documentclass[a4paper]{scrartcl}
\usepackage[ngerman]{babel}
\usepackage[utf8]{inputenc}
\usepackage[T1]{fontenc}
\usepackage{lmodern}
\usepackage{amssymb}
\usepackage{amsmath}
\usepackage{enumerate}
\usepackage{pgfplots}
\usepackage{scrpage2}\pagestyle{scrheadings}
\usepackage{tikz}
\usetikzlibrary{patterns}

\newcommand{\titleinfo}{Hausaufgaben zum 7. Juni 2012}

\title{\titleinfo}
\author{Elena Noll, Sven-Hendrik Haase, Arne Struck}
\date{\today}
\ihead{EN, SHH, AS}
\chead{\titleinfo}
\ohead{\today}
\setheadsepline{1pt}
\setcounter{secnumdepth}{0}
\newcommand{\qed}{\quad \square}

\begin{document}
\maketitle
\notag
\section{1.}
\subsection{(i)}

\[\int \frac{x+1}{x^2 -x-6}\ dx\]\\
0-Wertbestimmung des Nenners (Bestimmung der rationalen Nullstellen):
\[0=x^2 -x-6\]\\
pq-Formel:
\begin{align}
p=-1\ q=-6\\
x_{1/2}&=\frac{1}{2}\pm \sqrt{\Big(\frac{1}{2}\Big)^2 +6}\\
  &=\frac{1}{2}\pm \frac{5}{2}\\
	\Rightarrow & x_1 =-2 \land x_2 =3\\
	\Rightarrow & x^2-x-6=(x-3)(x+2)
\end{align}
\begin{align}
\frac{x+1}{x^2 -x-6}&= \frac{A}{x-3}+\frac{B}{x+2}\\
	&=\frac{A(x+2)+B(x-3)}{(x-3)(x+2)}\\
	&=\frac{Ax+2A+Bx-3B}{x^2 -x-6}\\
	&=\frac{(A+B)x+2A-3B}{x^2 -x-6}
\end{align}
Daraus folgt:
\begin{align}
A+B &= 1 \Leftrightarrow A=1-B\\
2A-3B &= 1\\
\Rightarrow 1&=2(1-B)-3B\\
	\Leftrightarrow 1&=2-5B\\
	\Leftrightarrow 1&=5B\\
	\Leftrightarrow \frac{1}{5} &= B\\
	\Rightarrow A&=-\frac{1}{5}+1\\
	\Rightarrow A&=\frac{4}{5}
\end{align}
A und B einsetzen:
\begin{align}
\frac{x+1}{x^2 -x-6}&=\frac{4}{5}\cdot \frac{1}{x-3}+\frac{1}{5}\cdot\frac{1}{x+2}\\
\Rightarrow	\int \frac{x+1}{x^2 -x-6}\ dx &=\frac{4}{5}\int\frac{1}{x-3}\ dx+\frac{1}{5}\int\frac{1}{x+2}\ dx\\
	&=\frac{4}{5}\ln |x-3|+\frac{1}{5}\ln |x+2|
\end{align}
Probe:
\begin{align}
\Big(\frac{4}{5}\ln |x-3|+\frac{1}{5}\ln |x+2|\Big)'&= \frac{4}{5}\Big(\ln |x-3|\Big)'+\frac{1}{5}\Big(\ln |x+2|\Big)'\\
	&=\frac{4}{5}\cdot\frac{1}{x-3}+\frac{1}{5}\cdot\frac{1}{x+2}\\
	&=\frac{4}{5}\cdot\frac{x+2}{(x-3)(x+2)}+\frac{1}{5}\cdot\frac{x-3}{(x+2)(x-3)}\\
	&=\frac{4(x+2)+x-3}{5(x^2 -x-6)}\\
	&=\frac{5x+5}{5(x^2 -x-6)}\\
	&=\frac{5(x+1)}{5(x^2 -x-6)}\\
	&=\frac{x+1}{x^2 -x-6}
\end{align}
\subsection{(ii)}
\begin{align}
\int \frac{2x+1}{x^2 -4x+4}\ dx &= \int \frac{2x-4 +1 +4}{x^2 -4x+4}\ dx\\
	&=\int \frac{2x-4}{x^2 -4x+4}\ dx +\int \frac{5}{x^2 -4x+4}\ dx\\
	&=\ln |x^2 -4x+4| + 5\int \frac{1}{x^2 -4x+4}\ dx\\
	&=\ln |(x-2)^2 | + 5\int \frac{1}{(x-2)^2}\ dx\\
\end{align}
Nebenrechnung:\\
Substitution:\\
Sei: \(u=x-2\)\\
\(\Rightarrow \ u'=1\ \land \ du=dx\)\\
\begin{align}
\int \frac{1}{(x-2)^2}\ dx &= \int \frac{1}{u^2}\ du\\
	&=\int u^{-2}\ du\\
	&= -u^{-1}\\
	&= -\frac{1}{(x-2)}
\end{align}
\begin{align}
\Rightarrow \int \frac{2x+1}{x^2 -4x+4}\ dx &= \ln|(x-2)^2 |+5 \int \frac{1}{(x-2)^2}\\
	&=\ln|(x-2)^2 |- \frac{5}{(x-2)}\\
\end{align}
\newpage
Probe:\\
\begin{align}
\Big(\ln|(x-2)^2 |- \frac{5}{(x-2)}\Big)'&=(\ln |(x-2)^2|)'-\Big(\frac{5}{(x-2)}\Big)'\\
	&=\frac{2x-4}{(x^2 -4x+4)}-5\Big(\frac{1}{(x-2)}\Big)'|Sei: u=(x-2)\\
	&=\frac{2x-4}{(x^2 -4x+4)}-5\Big(\frac{1}{u}\Big)'\\
	&=\frac{2x-4}{(x^2 -4x+4)}-5\cdot-\frac{1}{u^2}\\
	&=\frac{2x-4}{(x^2 -4x+4)}+\frac{5}{(x-2)^2}\\
	&=\frac{2x-4}{(x^2 -4x+4)}+\frac{5}{(x^2 -4x+4)}\\
	&=\frac{2x+1}{(x^2 -4x+4)}
\end{align}
\subsection{(iii)}
\begin{align}
\int \frac{4x+1}{x^2+4x+8}\ dx&=2\int \frac{2x+\frac{1}{2}}{x^2+4x+8}\ dx\\
 &=2\int \frac{2x+4+\frac{1}{2}-4}{x^2+4x+8}\ dx\\
 &=2\int \frac{2x+4}{x^2+4x+8}\ dx+\int \frac{\frac{1}{2}-4}{x^2+4x+8}\ dx\\
 &=2\ln |x^2+4x+8| -\frac{7}{2}\int \frac{1}{x^2+4x+8}\ dx\\
 &=2\ln |x^2+4x+8| -\frac{7}{2}\int \frac{1}{(x+2)^2+4}\ dx\\
 &=2\ln |x^2+4x+8| -\frac{7}{2*4}\int \frac{1}{\Big(\frac{x+2}{2}\Big)^2+1}\ dx\\
\end{align}
\newpage
\[Substitution\ von:\ \int \frac{1}{\Big(\frac{x+2}{2}\Big)^2+1}\ dx\]\\
Sei:\\
\(u=\frac{x+2}{2}\)\\ 
\(\Rightarrow u'=\frac{1}{2}\ \land\ dx=2\ du\)\\
\[2\int \frac{1}{u^2+1}\ dx =2\arctan(u)\]\\
\[Resubstitution: 2\arctan \Big(\frac{x+2}{2}\Big)\]
\begin{align}
\Rightarrow \int \frac{4x+1}{x^2+4x+8}\ dx&=2\ln(x^2+4x+8) -\frac{7\cdot 2}{8}\int\frac{1}{\Big(\frac{x+2}{2}\Big)^2+1}\ dx\\
	&=2\ln(x^2+4x+8) -\frac{7}{4}\arctan \Big(\frac{x+2}{2}\Big)\\
\end{align}
Probe:\\
\begin{align}
\Big(2\ln |x^2+4x+8| -\frac{7}{4}\arctan\Big(\frac{x+2}{2}\Big)\Big)'&=\Big(2\ln |x^2+4x+8|\Big)'-\frac{7}{4}\Big(\arctan\Big(\frac{x+2}{2}\Big)'\\
	&=2\cdot\bigg(\frac{2x+4}{x^2+4x+8}\bigg) - \frac{7}{4}\cdot\Bigg(\frac{1}{\Big(\frac{x+2}{2}\Big)+1}\Bigg)\\
	&=\frac{4x+8}{x^2+4x+8}-\frac{7}{(x+2)^2+4}\\
	&=\frac{4x+8}{x^2+4x+8}-\frac{7}{x^2+4x+8}\\
	&=\frac{4x+1}{x^2+4x+8}
\end{align}

\section{2.}
\subsection{a)}
\centerline{
\begin{tikzpicture}[scale=1.2]
  \begin{axis}[xmin=0, xmax=20,
    xlabel=$x$,
    ylabel={$f(x) = e^{-x}$}
  ]
    \addplot[smooth,domain=0:20,samples=40]{e^(-x)};
  \end{axis}
\end{tikzpicture}
}
Keine Wendepunkte vorhanden.\\

\centerline{
\begin{tikzpicture}[scale=1.2]
  \begin{axis}[xmin=0, xmax=40,
    xlabel=$x$,
    ylabel={$g(x) = \frac 1 {1+x}$}
  ]
    \addplot[smooth,domain=0:40,samples=40]{1/(1+x)};
  \end{axis}
\end{tikzpicture}
}
Keine Wendepunkte vorhanden.\\

\centerline{
\begin{tikzpicture}[scale=1.2]
  \begin{axis}[xmin=0, xmax=20,
    xlabel=$x$,
    ylabel={$h(x) = \frac 1 {1+x^2}$}
  ]
    \addplot[smooth,domain=0:20, samples=40]{1/(1+x^2)};
    \addplot[color=red,mark=*] coordinates {(1/sqrt(3), 3/4)};
  \end{axis}
\end{tikzpicture}
}
Wendepunkt bei \(x=\frac 1 {\sqrt{3}}\)\\


\newpage
\subsection{b)}
um den Flächeninhalt der Funktionen zu berechnen, werden Beträge uneigentlicher Integrale benötigt, daher:
\subparagraph{für \(f(x)=e^{-x}\):}
\begin{align}
\lim_{a\rightarrow \infty}\int_0^a f(x)\ dx &=\lim_{a\rightarrow \infty} \int_0^a e^{-x}\ dx\\ 
  &=\Big[e^{-x}\Big]_0^a &|für\ a\rightarrow\infty\\
	&=e^{-a}-e^0\\
	&=e^{-a}-1\\
	&=-1
\end{align}
\(\Rightarrow\) Der Flächeninhalt von f(x) ist gleich dem Betrag von -1, also 1.
\subparagraph{für \(g(x)=\frac{1}{x+1}\):}
\begin{align}
\lim_{b\rightarrow \infty} \int_0^b g(x)\ dx&=\lim_{b\rightarrow \infty} \int_0^b \frac{1}{x+1}\ dx\\
	&=\Big[\ln |x+1|\Big]_0^b &|für\ b\rightarrow\infty\\
	&=\ln |b|-\ln |0|\\
	&=\infty - (-\infty)\\
	&=\infty
\end{align}
\(\Rightarrow\) g(x) hat keinen endlichen Flächeninhalt.
\subparagraph{für \(h(x)=\frac{1}{x^2+1}\):}
\begin{align}
\lim_{c\rightarrow \infty} \int_0^c h(x)\ dx&=\lim_{x\rightarrow \infty} \int_0^c\frac{1}{x^2+1}\ dx\\
	&=\Big[\arctan(x)\Big]_0^c &|für c\rightarrow\infty\\
	&=\arctan(c)-\arctan(0)\\
	&=\frac{\pi}{2}-0\\
	&=\frac{\pi}{2}
\end{align}
\(\Rightarrow\) Der Flächeninhalt von h(x) beträgt \(\frac{\pi}{2}\).

\subsection{c)}
\centerline{
\begin{tikzpicture}[scale=1.5]
  \begin{axis}[xmin=-1, xmax=1,
    xlabel=$x$,
    ylabel={$h(x) = \frac 1 {\sqrt{1-x^2}}$}
  ]
    \addplot[smooth,domain=-2:2,samples=1000]{1/(sqrt(1-x^2))};
  \end{axis}
\end{tikzpicture}
}
Der Flächeninhalt von \(f(x)\) ist der Betrag von \(\int_{-1}^1 f(x)\ dx\).
\begin{align}
\int_{-1}^1 \frac{1}{\sqrt{1-x^2}}\ dx &= \Big[\arcsin(x)\Big]_{-1}^1\\
  &=\arcsin(1)-\arcsin(-1)\\
	&=\pi
\end{align}
\(\Rightarrow\) Der  Flächeninhalt von \(f(x):[-1,1]\rightarrow \mathbb{R}\) ist \(\pi\).

\newpage
\section{3.}
\(\int_0^1 \sin(x)\ dx\) näherungsweise bestimmen unter Verwendung der Trapezregel 
\\für \(n=4, n=5, n=10\).
\subparagraph{\(n=4\):}
\begin{align}
\int_0^1 \sin(x)\ dx &\approx\Big(\sin(0)+2\sin\Big(\frac{1}{4}\Big)+2\sin\Big(\frac{2}{4}\Big)+2\sin\Big(\frac{3}{4}\Big)+sin(1)\Big)\\
	&\approx0,4573
\end{align}
\subparagraph{\(n=5\):}
\begin{align}
\int_0^1 \sin(x)\ dx &\approx\Big(\sin(0)+2\sin\Big(\frac{1}{5}\Big)+2\sin\Big(\frac{2}{5}\Big)+2\sin\Big(\frac{3}{5}\Big)+2\sin\Big(\frac{4}{5}\Big)+\sin(1)\Big)\\
	&\approx0,45816
\end{align}
\subparagraph{\(n=10\):}
\begin{align}
\int_0^1 \sin(x)\ dx \approx &\Big(\sin(0)+2\sin\Big(\frac{1}{10}\Big)
+2\sin\Big(\frac{2}{10}\Big)+2\sin\Big(\frac{3}{10}\Big)+2\sin\Big(\frac{4}{10}\Big)\\
	&+2\sin\Big(\frac{5}{10}\Big)+2\sin\Big(\frac{6}{10}\Big)+
	2\sin\Big(\frac{7}{10}\Big)+2\sin\Big(\frac{8}{10}\Big)+2\sin\Big(\frac{9}{10}\Big)\\
	&+\sin(1)\Big)\\
	\approx &0,45931
\end{align}

\newpage
\section{4.}
\subsection{a)}
\[f(t)=9te^{-\frac 1 3 t}\]
Ableitungen berechnen:
\begin{align}
f'(t) &= 9t(-\frac 1 3 e^{-\frac 1 3 t}) + 9e^{-\frac 1 3 t} \\
      &= -3te^{-\frac 1 3 t} + 9e^{-\frac 1 3 t} \\
      &= e^{-\frac 1 3 t} (-3t+9) \\
f''(t) &= (-3te^{-\frac 1 3 t})' + (9e^{-\frac 1 3 t})' \\
       &= -3t (-\frac 1 3 e^{-\frac 1 3 t}) + (-3)e^{-\frac 1 3 t} + 9(-\frac 1 3 e^{-\frac 1 3 t}) \\
       &= te^{-\frac 1 3 t} - 6e^{-\frac 1 3 t} \\
       &= e^{-\frac 1 3 t} (t-6)
\end{align}

Maximum finden:
\begin{align}
 f'(t) &= 0 \\
\Leftrightarrow e^{-\frac 1 3 t}(-3t+9) &= 0 \\
\Leftrightarrow -3te^{-\frac 1 3 t} + 9e^{-\frac 1 3 t} &= 0 \\
\Leftrightarrow 9e^{-\frac 1 3 t} &= 3te^{-\frac 1 3 t} \\
\Leftrightarrow 3 &= t
\end{align}
Test für Maximum/Minimum mittels 2. Ableitung:
\[e^{-\frac 1 3 \cdot 3}(3-6) \approx -1.1\]
Ergebnis ist negative, also ist Maximum gefunden. \\
Das Maximum von ca. 9.93 Milligram pro Liter liegt bei \(t=3\).
\subsection{b)}
Berechnen des unbestimmten Integrals:
\begin{align}
f(t) &= 9te^{-\frac 1 3 t} \\
F(t) &= -3e^{-\frac 1 3 t} \cdot 9t - \int-3e^{-\frac 1 3 t} \cdot t\ dt \\
     &= -27te^{-\frac 1 3 t} + 27(-3)e^{-\frac 1 3 t} \\
     &= -27te^{-\frac 1 3 t} -81e^{-\frac 1 3 t} \\
     &= -27e^{-\frac 1 3 t} (t+3)
\end{align}
Jetzt Mittelwert mittels \(m=\frac 1 {b-a} \int^b_a f(x)\ dx \) berechnen:
\begin{align}
 &\frac 1 {12-0}[-27e^{-\frac 1 3 t} (t+3)]_0^{12} \\
=& \frac 1 {12-0}\Big(-27e^{-\frac 1 3 12}(12 + 3)-(-27e^{-\frac 1 3 0}(0 + 3))\Big) \\
\approx& 6.131
\end{align}
Der Mittelwert beträgt ca. 6.131.

\subsection{c)}
Finden des Wendepunkts:
\begin{align}
f''(t) &= 0 \\
te^{-\frac 1 3 t} - 6e^{-\frac 1 3 t} &= 0 \\
te^{-\frac 1 3 t} &= 6e^{-\frac 1 3 t} \\
t &= 6
\end{align}
\centerline{
\begin{tikzpicture}[scale=1.5]
  \begin{axis}[xmin=0, xmax=24,
    xlabel=$x$,
    ylabel={$f(x) = 9t \cdot e^{-\frac 1 3 t}$}
  ]
    \addplot[smooth,domain=0:24,samples=100]{9*x*e^(-1*x/3)};
    \addplot[color=red,mark=*] coordinates {(6, 7.308105295)};
  \end{axis}
\end{tikzpicture}
}
Der gesuchte Punkt des stärksten Abbaus (Wendepunkt) liegt bei \(x = 6\).

\newpage
\section{5.}
\subsection{a)}
\subsection{b)}
\[\int\cos\Big(\sqrt{\frac{x}{3}+1}\Big)\ dx\]\\
Substitution:\\
Sei:\\
\(u=\frac{x}{3}+1\)\\
\(\Rightarrow u'=\frac{1}{3}\ \land \ dx=3du\)\\
\[3\int\cos(\sqrt{u})\ du\]\\
Substitution:\\
Sei:\\
\(v=\sqrt{u}\)\\
\(\Rightarrow v'=\frac{1}{2\sqrt{u}}=\frac{1}{2v}\ \land \ du=2v\ dv\)\\
\[2\cdot 3\int\cos(v)v\ dv\]\\
\begin{align}
\int\cos(v)v\ dv&=\sin(v)v-\int\sin(v)\ dv\\
	&=\sin(v)v-\cos(v)
\end{align}
Resubstitution:
\begin{align}
\sin(v)v-\cos(v)&=\sin(\sqrt{u})\cdot\sqrt{u}-\cos\sqrt{u}\\
	&=\sin\Big(\sqrt{\frac{x}{3}+1}\Big)\ \cdot\sqrt{\frac{x}{3}+1}-\cos\Big(\sqrt{\frac{x}{3}+1}\Big)\\
	\Rightarrow \int\cos\Big(\sqrt{\frac{x}{3}+1}\Big)\ dx &=6\Bigg(\sin\Big(\sqrt{\frac{x}{3}+1}\Big)\ \cdot\sqrt{\frac{x}{3}+1}-\cos\Big(\sqrt{\frac{x}{3}+1}\Big)\Bigg)
\end{align}
\subsection{c)}
\subsection{d)}
\[\int\frac{3x+2}{x^2-10x+25}\ dx\]
\begin{align}
\frac{3x+2}{x^2-10x+25}&=\frac{3x+2}{(x-5)^2}\\
	&=\frac{A}{(x-5)^2}+\frac{B}{x-5}\\
	&=\frac{A+B(x-5)}{(x-5)^2}\\
	&=\frac{Bx+A-5B}{(x-5)^2}
\end{align}
Daraus folgt:\\
\(B=3\\
A-5B=2\\
A-15=2\\
A=17\)
\begin{align}
\Rightarrow\int\frac{3x+2}{(x-5)^2}\ dx&=\int\frac{17}{(x-5)^2}\ dx+\int\frac{3}{(x-5)^2}\ dx\\
	&=17\int\frac{1}{(x-5)^2}\ dx+3\int\frac{1}{x-5}\ dx\\
	&=3\ln |x-5|-\frac{17}{x-5}
\end{align}

\end{document}


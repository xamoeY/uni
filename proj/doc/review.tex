\documentclass[11pt,twoside,a4paper]{article}
\usepackage{amsmath}
\usepackage{mathtools}
\usepackage{graphicx}
\graphicspath{ {/home/alisa/uni-sven/proj/pics/} }

\begin{document}
\title{Development and programming of a micro processor}
\author{Alisa Dammer and Sven-Hendrik Haase}
\maketitle

\section{Introduction}
\subsection{General information: How does CPU work}
A central processing unit (CPU) is a heart of a computer. It presents arithmnetical logic, logic operations and input-output operations. Nowadays one computer can have more than one CPU, called multiprocessors. But in this project we have concentrated our attention on one single microprocessor.\\
In order to work properly and execute commands (perform certain operations), CPU needs to go through several stages:
\begin{enumerate}
	\item[1.] Fetch: read the program, that is stored in memory as instructions. (We implemented an Assambeler, that translates the program from "human" language to instructions. More about it in  section "CPU implementation"). To keep an eye on right order of instructions Program Counter (PC) keeps the address of the next executable instruction.
	\item[2.] Decode: Instruction is a representation of a assembly command in 0 and 1. During this stage every executable instruction is broken up into several parts. The amount and length of these pieces depend on the type of instruction (more details in the next section). 
	\item[3.] Execute: According to what kind of instruction has to be performed, different units of the CPU can be involved. For example: ALU and two registers, or CU and register etc. Normally modern CPUs have overflow flags, that are "activated", if the output is too big for restricted CPU (by "restricted" we mean, that all CPUs are by design implemented to deal with limited values).
	\item[4.] Writeback: During this stage the result of an instruction is "returned" to some memory part, so that it could be used later on.
\end{enumerate}

All these 4 steps repeat for every instruction until the "stop" instruction is reached (logical end of the program), which is normally refers to "no action" or "terminate". As was mentioned above in Fetch-stage, PC holds the address of the next executable instruction, incrementing each time the instruction is executed. Some of control instructions like "jump" change PC according to needs of the program, for example by implementing a cycle.\\
\subsection{Main components of a simple CPU}
Every CPU's design starts with Instruction Set Architecture (ISA) definition, that describes what particular prozessor is capable of: general specification (word length, instruction length), instruction specifications and opcodes (operations, that can be presented in binary form), type of memory that is used (number of special purpose and general purpose registers), and other specific infromation if needed. \\
Apart from ISA all microprozessors have certain units like: Program Counter (PC), Control Unit (CU), Aruthmetic Logic Unit (ALU), Registers (Register Bank), Buses.\\
\begin{center}
\includegraphics[scale=0.5]{simpleCPU3.png}
\end{center}

However our implementation differs from existing ones. This will be shown in details in next section.\\ 

\newpage
\section{CPU implementation}
\subsection{Instruction Set Architecture}
As was mentioned in the previous section, ISA defines all specifications of a CPU. In our work we decided to stop on 16bit instructions, that have the same lenght as word, so we didn't have to think about instructions consisting of more than one word, which also makes Instruction Unit (our "instruction breaker") easier to implement.
\begin{verbatim}
general specifications:
	16 bit instructions
	16 bit words
	14 registers:
		8 general purpose registers
		6 special purpose registers
\end{verbatim}
Our cpu has only one type of instructions, also we don't have special flags. Instead of special flags we use special purpose registers and certain instructions. As already was said: we have implemented pretty simple microprocessor.
\begin{center}
\includegraphics[scale=0.7]{instruction1.png}
\end{center}
Because we have set the lenght of opcode to 4 bits, we could have only 16 = 2²(to the power of 2) But we have 16 opcodes for "real" instructions and 18 "pseudo" instructions without opcode, but they are replaced with composition of actual instructions. For example, instruction "add" opeartes with two registers, saving result of the operation to the left (or first) register:
\begin{verbatim}
movi a 5
movi b 4
add a b
stop
\end{verbatim}
This program will result in:
\begin{verbatim}
program start
PC0 => movi a 7
PC1 => movi b 4
PC2 => add a b
program end
--------------------
registers:
zero: 0
pc: 3
cmp_result: 0
jmp_next: 0
tmp1: 0
tmp2: 0
a: 11
b: 4
c: 0
d: 0
e: 0
f: 0
g: 0
h: 0
\end{verbatim}
Here as first example all registers, that we have are shown, later on only registers, that get some value will be shown. "Program end" - is our "stop" instruction (Mostly known as HALT). So the result is that register a gets the value 11 = 7 + 4. We didn't implement any special output register, cause we can always store (or move to desirable "output"-register) data manually.\\
Both only actual instruction list and all-instruction list can be found in our isa.txt file.\\
Pseudo instructions can be represented by composition of actual instructions. Some of them consist of two instructions, some of them contain up to 5 instructions. For example:
\begin{verbatim}
movi a 7
addi a 5
stop
\end{verbatim}
The original program consist of 3 instructions, but it results in program with 4, because addi is a pseudo instruction replaced with movi and add instructions.
\begin{verbatim}
program start
PC0 => movi a 7
PC1 => movi tmp1 5
PC2 => add a tmp1
program end
--------------------
\end{verbatim}
Another example for a pseudo instruction would be "modulo" program, 45 mod 10 = 5
\begin{verbatim}
movi a 45
movi b 9
mod a b
stop
\end{verbatim}
But it will result in:
\begin{verbatim}
program start
PC0 => movi a 45
PC1 => movi b 10
PC2 => mov tmp1 a
PC3 => div a b
PC4 => mul a b
PC5 => sub tmp1 a
PC6 => mov a tmp1
program end
--------------------
registers:
pc: 7
tmp1: 5
a: 5
b: 10
\end{verbatim}
Apart from instructions (opcodes) and general information ISA gives information about REGISTERS: general and special purpose.
\begin{verbatim}
registers:
	zero 		000000 # always returns value 0
	
	pc			000001 # program counter
	cmp_result  000010 # result of comparison as defined below
	jmp_next    000011 # je will go here if the comparison succeeds
	tmp1        000100 # temporary register to be used by compiler
	tmp2		000101 # another temporary register

	# general purpose registers
	a       	000110
	b       	000111
	c      	 	001000
	d       	001001
	e       	001010
	f       	001011
	g      	 	001100
	h       	001101
\end{verbatim}
In our project we decided not to implement both ALU and CU, but unite them in control Unit. Execution of a programm in our cese goes through following stages:
\begin{enumerate}
	\item[1.] Encode - assembly programm is "translated" with the help of written Assembler to binary string. Outout file gets an extention ".o" and consist of N 16bit strings, where N is a number of "actual instructions" (here, if pseudo instruction is used in the programm, it is translated into several actual instructions for further executing). So, as the resukt we get Machine Code.
	\item[2.] Decode - every instruction in output file is split into 3 parts: opcode, operand 1 and ooperand 2 according to number of bits specified in ISA. So, as the result of this stage we get executable instructions, where we know what operations with what operands should be performed. 
	\item[3.] Execution - actual performing of operations with given values, using memory and registers according to the originally written .asm programm. As the result of executing a program we get the result "stored" in desirable memory location or register (if specified in the program), or by default in the last "mentioned" left register.\\
Our CPU can be presented as following schema:
\end{enumerate} 
\begin{center}
\includegraphics[scale=0.7]{ourCPU.png}
\end{center}

\newpage
\subsection{Assembler}
As was told earlier, our first stage of the program execution is encoding of the program for futher use. This happens in assempler.py unit in following order:
\begin{enumerate}
	\item[1.] First, the input file with .asm extention is opend into source file for reading, splited into root and extenion. And binary file is created with extention .o.
	\item[2.] This binary file for every code line	"translates" human code into bit-string, that machine can interprete.
	\item[3.] Depending on type of instruction, whether it is an actual or pseudo instruction command will be converted into 1 or more bit-strings. (examples of pseudo instructions are given above). Every bit-string is added to the binary file. As the result we get a list containing N bit-strings.
\end{enumerate}
For example with modulus binary file will look like this:
\begin{verbatim}
alisa@comrade ~/u/proj> python2 assembler.py programs/test.asm
['0100000110101101', '0100000111001001', '0011000100000110', '1000000110000111', '0111000110000111', '0110000100000110', '0011000110000100', '0000000000000000']
\end{verbatim} 

\end{document}


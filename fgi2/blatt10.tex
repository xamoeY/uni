\documentclass[a4paper,12pt]{scrartcl}

\usepackage[utf8]{inputenc}
\usepackage[ngerman]{babel}
\usepackage{multicol}
\usepackage{scrpage2}\pagestyle{scrheadings}
\usepackage{graphicx} 
\usepackage{mathtools}
\usepackage{amssymb}
\usepackage{pgfplots}
\usetikzlibrary{arrows,automata,positioning,patterns,shapes,petri}

\ihead{Blatt 10}
\chead{Elena Noll, Sven-Hendrik Haase}
\ohead{\today}
\pagestyle{scrheadings}
\setheadsepline{1pt}
\setcounter{secnumdepth}{0}

\begin{document}

\section{Übungsaufgabe 10.3}

\subsection{1.}
Menge aller T-Invarianten:
\[
	\left\{ a \begin{pmatrix} 1 \\ 0 \\ 0 \\ 1 \end{pmatrix} +
		    b \begin{pmatrix} 0 \\ 1 \\ 1 \\ 0 \end{pmatrix} \right\}
\]
mit \(a, b \in \mathbb{N}\)

\subsection{2.}
Wirkunsmatrix:
\[
	\bordermatrix{
	   &  a	&  b &  c &  d \cr
	p1 &  1 & -2 &  2 &  -1 \cr
	p2 & -1 &  3 &  -3 &  1 \cr
	p3 &  0 &  -1 & 1 &  0 \cr
	p4 &  0 &  1 & -1 &  0 \cr}
\]
Anfangsmarkierung \(m_0\): \( \begin{pmatrix} 3 \\ 3 \\ 2 \\ 2 \end{pmatrix} \) \\
Schaltfolge:
\[ 
	\begin{pmatrix} 3 \\ 3 \\ 2 \\ 2 \end{pmatrix} \xrightarrow{a}
	\begin{pmatrix} 4 \\ 2 \\ 2 \\ 2 \end{pmatrix} \xrightarrow{d}
	\begin{pmatrix} 3 \\ 3 \\ 2 \\ 2 \end{pmatrix} \xrightarrow{b}
	\begin{pmatrix} 1 \\ 6 \\ 1 \\ 3 \end{pmatrix} \xrightarrow{c}
	\begin{pmatrix} 3 \\ 3 \\ 2 \\ 2 \end{pmatrix} \xrightarrow{b}
	\begin{pmatrix} 1 \\ 6 \\ 1 \\ 3 \end{pmatrix} \xrightarrow{c}
	\begin{pmatrix} 3 \\ 3 \\ 2 \\ 2 \end{pmatrix}
\]

\section{Übungsaufgabe 10.4}
\subsection{1.}
Da beim Schalten der Transition einer Falle zwar immer eine Markierung aus der 
Stelle entfernt wird, aber im gleichen Schritt auch wieder eine darin generiert
wird, ist es unmöglich, eine Falle von einer Markierung zu befreien, wenn diese
in der Anfangsmarkierung eine hat.

\subsection{2.}
Da es nicht möglich ist, eine Falle von außen mit einer Markierung zu versehen
und die Falle ohne Markierung nicht schaltet, muss die Falle von Anfang an eine
Markierung haben.

\subsection{3.}
Menge der Fallen von \(N_{10.4}\):
\begin{equation*}
\begin{split}
	\Big\{ &\{p5\}, \{p3, p4\}, \{p1, p2, p3, p4, p5\}, \{p1, p2, p3, p4\},\\
		   &\{p1, p3, p4\}, \{p2, p3, p, p5\}, \{p3, p4, p5\}, \{p1, p3, p4, p5\} \\
		   &\{p1, p2, p3, p5\}, \{p2, p4, p5\} \Big\}
\end{split}
\end{equation*}

\subsection{4.}
Anfangsmarkierung \(m_0\): \( \begin{pmatrix} 1 \\ 0 \\ 0 \\ 1 \\ 1 \end{pmatrix} \) \\
Pfade:
\begin{align*}
	\sigma^a &= m_0 \xrightarrow{a} m' \\
	\sigma^b &= m_0 \xrightarrow{a} m' \xrightarrow{b} m'' \\
	\sigma^{a'} &= m_0 \xrightarrow{a'} m' \\
	\sigma^c &= m_0 \xrightarrow{a} m' \xrightarrow{c} m'' \\
	\sigma^{c'} &= m_0 \xrightarrow{a'} m' \xrightarrow{c'} m''
\end{align*}

\end{document}

\documentclass[a4paper,fleqn]{scrartcl}
\usepackage[ngerman]{babel}
\usepackage[utf8]{inputenc}
\usepackage[T1]{fontenc}
\usepackage{lmodern}
\usepackage{amssymb}
\usepackage{amsmath}
\usepackage{enumerate}
\usepackage{scrpage2}
\usepackage{tikz}

\ihead{AD Übung 3}

\ohead{Alisa Dammer, Elena Noll, Sven-Hendrik Haase}
\pagestyle{scrheadings}
\setheadsepline{1pt}
\newcommand{\qed}{\quad \square}

\begin{document}
\subsection*{Aufgabe 1}
\subsubsection*{a)}
\begin{enumerate}
\item \([6, 18, 14, 19, 15, 21, 20]\)
\item \([6, 18, 14, 19, 15, 20, 21]\)
\item \([6, 14, 18, 19, 15, 20, 21]\)
\item \([6, 14, 15, 19, 18, 20, 21]\)
\item \([6, 14, 15, 18, 19, 20, 21]\)
\end{enumerate}

\subsubsection*{b)}
\begin{verbatim}
search(x, liste)
    i = 1
    j = liste.länge
    if x > liste[liste.länge - 1] + liste[liste.länge] oder
       x < liste[1] + liste[2]
        return -1 // Fehlercode
    while liste[i] + liste[j] != x
        if i == j
            return -2 // Fehlercode
        if liste[i] + liste[j] < x
            i += 1
        elseif liste[i] + liste[j] > x
            j -= 1
    return Pair(liste[i], liste[j])
\end{verbatim}

Die beiden Laufindizes \(i, j\) werden so gewählt, dass sie auf den Anfang und das Ende
des Arrays zeigen. Dies ist notwendig, denn dadurch ist gewährleistet, dass alle
möglichen Paare \((a, b)\), für die \(x = a + b\) gilt, eingeschlossen werden.
\(a\) ist der Wert an der Stelle \(i\), \(b\) ist der Wert an der Stelle \(j\).
Zudem führt dies mit unserem Algorithmus zu einer Laufzeit von \(O(n)\).

Beginnend wird geprüft, ob \(x\) mit den Werten im Array überhaupt erreichbar 
ist, indem wir prüfen, ob \(x \leq Array[1] + Array[2]\) oder \(x \geq Array
[Array.länge - 1] + Array[Array.länge]\) ist. Sollte einer dieser Fälle zutreffen, 
so brechen wir mit einem Fehlerwert ab.

Wenn die vorhergegangene Prüfung keinen Fehlerwert zurückgibt,
wird geprüft, ob \(x \not= a + b\), denn solange ist das Ergebnis nicht erreicht.
Wenn das der Fall ist, wird geprüft, ob \(x \leq a + b\) oder \(x \geq a + b\).
Dies ist entscheidend dafür, wie unsere Laufindizes verändert werden. Wenn
\(x \leq a + b\), wissen wir, dass es nur Sinn macht, \(i\) zu erhöhen, denn
nur die Erhöhung einer der beiden Summanden führt zu einer Erhöhung der Summe.
Da \(b\) bereits den höchsten Wert hat, kann nur der Wert von \(a\) erhöht werden.
Andersherum entsprechend, wenn \(x \geq a + b\), muss \(j\) verringert werden,
damit die Summe somit geringer wird. Nach der Verränderung wird die Schleife
erneut ausgeführt und die Kondition evaluiert.
Wenn alle möglichen Summen getestet wurden und das Ergebnis dennoch nicht erreicht wurde,
können wir abbrechen und einen Fehlerwert zurückgeben, weil das Ergebnis mit 
den Werten des Arrays nicht erreicht werden kann.


\end{document}
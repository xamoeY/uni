\documentclass[a4paper]{scrartcl}
\usepackage[ngerman]{babel}
\usepackage[utf8]{inputenc}
\usepackage[T1]{fontenc}
\usepackage{lmodern}
\usepackage{amssymb}
\usepackage{amsmath}
\usepackage{enumerate}
\usepackage{scrpage2}
\usepackage{tikz}

\ihead{AD}
\chead{Aufgaben zum 22.10.12}
\ohead{Elena Noll, Sven-Hendrik Haase}
\pagestyle{scrheadings}
\setheadsepline{1pt}
\newcommand{\qed}{\quad \square}

\begin{document}
\section{Aufgabe}
\begin{enumerate}[a)]
 \item Geordnet nach absteigender Geschwindigkeit \\
 $
 10^{10} = O(1) \\
 \log(n^{10}) = O(\log N^{10}) \\
 \log^3(n) = O(\log^3 N) (log^3\ ist\ langsamer\ als\ log) \\
 \sqrt{n} = O(\sqrt{N}) \\
 n-\frac{n}{2} = O(N) \\
 n \log^2 (n) = O(N \log N) \\
 n^3 = O(N^3) \\
 2^{\log n} = O(2^{log N)} \\
 \left(\frac{3}{2}\right)^n = O\left(\left(\frac{3}{2}\right)^N\right) (die\ Rate\ des\ Wachstums\ bei\ log(N)\ ist\ geriner\ als\ bei\ N)\\
 n! = O(N!)
 $
 \item
 \begin{enumerate}[1.]
  \item 
  $n^k = O(n^l) $ für alle $ l\geq k$, Anwendung Regel 2: $ 10 \geq 1$
  \item
  $ \log_b(n) = \frac{\log(n)}{\log(b)} = \frac{1}{\log(b)} \cdot \log(n) = O(\log N)$
  \item Wenn man einsetzt, ergibt sich eine Hintereinanderausführung von Funktionen. Der Wert ist in beiden Fällen der Gleiche.
  Die Behauptung ist also wahr.
  \item Die Summe wird gegen \(n=\infty\) größer als O(n). Der Ausdruck ist falsch.
 \end{enumerate}
\item
Zwei Schleifen werden geschachtelt N-mal durchlaufen. \\
ALG1() = $O(N^2) \\$
i nähert sich quadratisch gegen N, damit ist die Laufzeit des Algorithmus logarithmisch. \\
ALG2() = $O(\log N) \\$
Die äußere Schleife ist linear, in der innerern Schleife halbiert sich j und somit ist diese logarithmisch. \\
ALG3() = $O(N\log N)$
\end{enumerate}
\section{Aufgabe}
\begin{enumerate}
 \item[b)]
 Erwartungswert:
 \[\sum_{x=1}^n x \cdot \frac{1}{n}\]
 
\end{enumerate}


\end{document}
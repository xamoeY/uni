\documentclass[12pt]{article}
\usepackage[utf8]{inputenc}
\usepackage{amssymb,amsmath}
\usepackage{hyperref}
\author{Claas Jaehrling, Sven-Hendrik Haase}
\title{RS1 HA zum 01.12.11}
\date{\today}
\begin{document}
\setcounter{secnumdepth}{0}
\maketitle

\section{Aufgabe 5.1}
26 Großbuchstaben sollen in einem zyklisch-einschrittigen Binärcode codiert werden.
Für 2 Zeichen benötigt man die Wortlänge 1:
\[\{0,1\}\]
\[4: \{00,01,11,10\}\]
\[8: \{000,010,110,100,101,111,011,001\}\]
\[16: \{0000,0100,1100,1000,1010,1110,0110,0010,\]
\[0011,0111,1111,1011,1001,1101,0101,0001\}\]
\[32: \{00000,01000,11000,10000,10100,11100,01100,00100,\]
\[00110,01110,11110,10110,10010,11010,01010,00010,\]
\[00011,01011,11011,10011,10111,11111,01111,00111,\]
\[00101,01101,11101,10101,10001,11001,01001,00001\}\]
Für 26 Codewörter benötigen wir ebenfalls die Wortlänge 5. Da der Code zyklisch
sein soll, nehmen wir nun die 6 überzähligen Wörter aus der Mitte heraus:
\[26: \{00000,01000,11000,10000,10100,11100,01100,00100,\]
\[00110,01110,11110,10110,10010,\]
\[10011,10111,11111,01111,00111,\]
\[00101,01101,11101,10101,10001,11001,01001,00001\}\]
Hiermit können die Großbuchstaben A-Z codiert werden:
\[A:00000, B:01000, C:11000, D:10000, E:10100, F:11100, G:01100,\]
\[H:00100, I:00110, J:01110, K:11110, L:10110, M:10010, N:10011,\]
\[O:10111, P:11111, Q:01111, R:00111, S:00101, T:01101, U:11101,\]
\[V:10101, W:10001, X:11001, Y:01001, Z:00001.\]

\section{Aufgabe 5.2}

\section{Aufgabe 5.3}
\subsection{(a)}
\subsection{(b)}

\section{Aufgabe 5.4}
\subsection{(a)}
\subsection{(b)}
\subsection{(c)}

\begin{align}
\end{align}
\end{document}

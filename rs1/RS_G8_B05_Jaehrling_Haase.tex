\documentclass[12pt]{article}
\usepackage[utf8]{inputenc}
\usepackage{amssymb,amsmath}
\usepackage{hyperref}
\author{Claas Jaehrling, Sven-Hendrik Haase}
\title{RS1 HA zum 01.12.11}
\date{\today}
\begin{document}
\setcounter{secnumdepth}{0}
\maketitle

\section{Aufgabe 5.1}
\begin{verbatim}
2:  {0, 1}
4:  {00, 01, 11, 10}
8:  {000, 001, 011, 010, 110, 111, 101, 100}
16: {0000, 0001, 0011, 0010, 0110, 0111, 0101, 0100,
     1100, 1101, 1111, 1110, 1010, 1011, 1001, 1000}
32: {00000, 00001, 00011, 00010, 00110, 00111, 00101, 00100, 01100, 01101,
     01111, 01110, 01010, 01011, 01001, 01000, 11000, 11001, 11011, 11010,
     11110, 11111, 11101, 11100, 10100, 10101, 10111, 10110, 10010, 10011,
     10001, 10000}

Jetzt können wir in der Mitte Paare wegnehmen:

30: {00000, 00001, 00011, 00010, 00110, 00111, 00101, 00100, 01100, 01101, 
     01111, 01110, 01010, 01011, 01001, 11001, 11011, 11010, 11110, 11111, 
     11101, 11100, 10100, 10101, 10111, 10110, 10010, 10011, 10001, 10000}
28: {00000, 00001, 00011, 00010, 00110, 00111, 00101, 00100, 01100, 01101,
     01111, 01110, 01010, 01011, 11011, 11010, 11110, 11111, 11101, 11100,
     10100, 10101, 10111, 10110, 10010, 10011, 10001, 10000}
26: {00000, 00001, 00011, 00010, 00110, 00111, 00101, 00100, 01100, 01101,
     01111, 01110, 01010, 11010, 11110, 11111, 11101, 11100, 10100, 10101,
     10111, 10110, 10010, 10011, 10001, 10000}




Jetzt können wir zuordnen:
A    00000
B    10000
C    10100
D    11100
E    01100
F    00100
G    00110
H    01110
I    11110
J    10110
K    10010
L    11010
M    01010
N    00010
O    00011
P    01011
Q    11011
R    10011
S    10111
T    00111
U    00101
V    01101
W    11101
X    10101
Y    10001
Z    00001
\end{verbatim}
 
%\begin{tabular}{l l l}
%	A & B & C & D & E & F & G & H & I & J & K & K L M N O P Q R S T U V W X Y Z
%\end{tabular}

\section{Aufgabe 5.2}
Mit Codewörtern: \{A, B, C, D\}: \{00, 01, 11, 10\} \\
Mit Codewörtern: \{A, B, C\}: \{00, 01, 10\} (kaputt, da 2 Flips von 01 - 10) \\
Da der rekursive Algorithmus symmetrische Ergebnisse liefert, versagt dieses Verfahren bei einer ungeraden Anzahl an Codewörtern. Wir können hier nicht einfach die beiden mittigen Werte eliminieren.

\section{Aufgabe 5.3}
\subsection{(a)}
\subsection{(b)}

\section{Aufgabe 5.4}
\subsection{(a)}
\subsection{(b)}
\subsection{(c)}

\begin{align}
\end{align}
\end{document}

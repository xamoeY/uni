\documentclass[a4paper,12pt]{article} 
\usepackage{german}
\usepackage[T1]{fontenc}
\usepackage[utf8]{inputenc}
\usepackage{graphicx}
\usepackage{float}
\usepackage[numbib]{tocbibind}
\usepackage{gensymb}
\usepackage{pgfplots}
\usepackage{tikz}
\begin{document}
\renewcommand\bibname{Referenzen}
\renewcommand\refname{Referenzen}

\begin{titlepage}
    \author{Alisa Dammer\\
    Sven-Hendrik Haase}
    \title{ParaQuest}
    \date{\today} 
    \maketitle
    \thispagestyle{empty}
\end{titlepage}

\tableofcontents

\newpage

\section{Einführung}
ParaQuest ist eine hochparallele Simulation von Kreaturen in einer 2D-Welt
mit einem simplen Kampf- und Kollisionssystem. Die technische Umsetzung
erfolgte mit C++11, cereal (für die Serialisierung) und MPI (für die Prozesskommunikation) sowie Qt (für die Visualisierung).

\section{Problemstellung}
Es soll eine Simulation erstellt werden, deren Berechnung per MPI auf
mehrere Prozessen verteilt stattfinden kann. Simulationsgegenstand ist hierbei
eine 2D-Kachelwelt voller Kreaturen, die sich bewegen können und kämpfen,
wenn mehrere von ihnen auf dem gleichen Feld stehen. Dabei soll ein einfaches
Kampfsystem mit Trefferpunkten, Angriffsstärke und Ausweichchance zum Einsatz
kommen.\\
Zudem soll es mehrere verschiedene Spezies von Kreaturen geben: Goblin, Hobbit, Orc, Elf, Dwarf, Human. Jede Spezies hat andere Werte der Attribute Strength und Agility. Diese Attribute sind entscheiden bei der Kampfsimulation.
Jede Kreature kann sich pro Welttick um ein Feld bewegen. Einige Felder sind
durch Steine gesperrt. Auf diesen Feldern kann sich keine Kreatur befinden.
Es soll zudem eine schicke Visualisierung der Welt erstellt.

\section{Lösungsansatz}
\subsection{Simulation}
Die Welt wird als sparse Matrix umgesetzt, da sich nicht in jedem Feld immer
eine Kreatur befindet. Jede Kreatur bekommt also einfach Ganzzahlkoordinaten
als Attribut sowie eine global eindeutige Id.\\
Am Anfang der Simulation wird die Welt im Hauptprozess populiert und dann an
die anderen Prozesse verteilt.
Nach jedem Tick wird von jedem Prozess der jeweils eigene Teil der Welt gedumped.
Dieser Dump dient als Resultat des Simulationsschrittes und wird später für
die Visualisierung benötigt.
\\\\
Eine Kreatur hat folgende simulationsrelevante Attribute:
\begin{verbatim}
	spezies
	position
	strength
	agility
	hitpoints
\end{verbatim}

Die Welt hat folgende simulationsrelevante Attribute:
\begin{verbatim}
	size
	creatures-list
\end{verbatim}

Das gesamte Programm wurde programmatisch mit C++11 umgestzt. Für die
Serialisierung und die Erstellung der Dumps kommt cereal zum Einsatz.
Für die Interprozesskommunikation wird MPI eingesetzt. Qt wird für die
Visualisierung verwendet.
\\\\
Die Simulation hat folgenden Ablauf:
\begin{enumerate}
	\item Welt im Hauptprozess bevölkern
	\item bevölkerte Welt an andere Prozesse aufteilen und senden
	\item in jedem Prozess die Welt für n Ticks simulieren
\end{enumerate}
Im Detail geschieht während der Simulation der Welt folgendes:
\begin{enumerate}
	\item alle Kreaturen bewegen
	\item Kreaturen an den Rändern an Nachbarn senden bzw. von Nachbarn empfangen
	\item Kollisionen finden und Kreaturen kämpfen lassen
	\item Verlierer und gesendete Kreaturen aus sich selber löschen
	\item eigene Welt dumpen
\end{enumerate}

Ein Dump der Welt hat die Aufgabe, als inkrementelles Simulationsresultat
des jeweiligen Simulationsschrittes zu dienen. Es wird nach folgendem Schema
benannt:
\begin{verbatim}
	<knotenname>-<knotenrank>_tick-<simulationsschritt>.log
\end{verbatim}
Der Inhalt besteht aus einer binär serialisierten Liste aller Kreaturen dieses
Prozesses und ihrer Attribute. Alle diese Logs werden später in der
Visualisierung wieder zu einem Gesamtresultat zusammengesetzt.
\\\\
Das Simulationsprogramm nimmt folgende Parameter:
\begin{itemize}
	\item -h zeigt die Hilfe an
	\item -r <seed> setzt einen gewählten Randomseed
	\item -s <size> setzt die Weltdimensionen nach size x size
	\item -c <count> setzt die maximale Anzahl der Startkreaturen
	\item -t <ticks> setzt die Anzahl der Simulationsschritte
	\item -o <count> setzt die Anzahl der gesperrten Felder 
\end{itemize}

\subsection{Visualisierung}
Die Visualisierung erlaubt es dem Benutzer per Schiebeelemente, sich
verschiedene Ausschnitte der Welt anzuschauen. Auf der rechten Seite
befindet sich der Schieberegler für den Zoomlevel, auf der unteren Seite
befindet sich ein Schieberegler für den Simulationsschritt. Man kann den
Ausschnitt bewegen, indem man klickt und zieht.
\begin{figure}[H]
	\centerline{\includegraphics[scale=0.8]{visualisierung0.png}}
	\caption{Visualisierung}
\end{figure}

\section{Parallelisierungsschema}
Die Welt wird zeilenweise aufgeteilt. D.h. eine Welt der Größe 100x100 wird
auf 5 Prozesse so aufgeteilt, dass jeder Prozess einen Bereich von 20x100 Felder
zugewiesen bekommt. Die Prozesse kommunizieren nach links und rechts mit ihren
direkten Nachbarn.

Zusätzlich zum eigenen Bereich bekommt jeder Prozess noch
eine Extraspalte des jeweils direkten Nachbarn links und rechts, um
bereichsübergreifende Kollisionsabfragen durchführen zu können. Kreaturen, die
aus dem eigenen Bereich herauslaufen, werden an den angrenzenden Nachbar
gesendet und aus dem eigenen Speicher gelöscht.

\section{Laufzeitmessungen}
Mit folgenden Parametern wurden die Messungen durchgeführt:
\begin{verbatim}
	./paraquest_simulation -r 10 -s 10000 -t 1000 -c 1000 -o 1000
\end{verbatim}
\begin{itemize}
	\item 1 Prozess:  285,8s
	\item 2 Prozesse: 74,4s
	\item 4 Prozesse: 23,34s
	\item 8 Prozesse: 11,9s
\end{itemize}

\section{Leistungsanalyse}
Bei der Leistungsanalyse ist besonders aufgefallen, dass am meisten Zeit
in der Kollisionsabfrage benötigt wird. Dies hängt damit zusammen, dass die
Kollisionsabfrage on \(O(n^2)\) erfolgt, während die Kampfschleife \(O(n)\)
Zeit benötigt.

Das Senden und Empfangen verwendet dank der dünnen Matrix einen sehr geringen
Teil der Laufzeit.

\section{Skalierbarkeit}
Bei der Skalierbarkeit ist uns aufgefallen, dass das Programm scheinbar
superlinear skaliert. Hierführ haben wir keine Erklärung. Bei mehr Prozessen
verhält sich das Programm annähernd linear.

\end{document}

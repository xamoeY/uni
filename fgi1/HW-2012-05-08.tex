\documentclass[a4paper]{scrartcl}
\usepackage[ngerman]{babel}
\usepackage[utf8]{inputenc}
\usepackage[T1]{fontenc}
\usepackage{lmodern}
\usepackage{amssymb}
\usepackage{amsmath}
\usepackage{enumerate}
\usepackage{scrpage2}\pagestyle{scrheadings}
\usepackage{tikz}
\usetikzlibrary{arrows,automata}

\newcommand{\titleinfo}{Hausaufgaben zum 08.05.2012}
\newcommand{\aufgabe}[1]{\item[\textbf{#1}]}

\title{\titleinfo}
\author{Arne Feil}
\date{\today}
\ihead{FGI1}
\chead{Arne Feil, Sven-Hendrik Haase, Christian Darsow-Fromm}
\ohead{\today}
\setheadsepline{1pt}
\newcommand{\qed}{\quad \square}
\newcommand{\terminal}[1]{\langle #1 \rangle }
\everymath{\displaystyle}
\begin{document}
%\maketitle
\begin{enumerate}

\aufgabe{5.2}
\underline{Behauptung}: Die Tiefe einer Formel ist echt kleiner als ihre Länge:
\[\forall F \in L_{AL} : 1+tiefe(F) \le |F| \]
\underline{Induktionsanfang}: TODO\\
\underline{Induktionsannahme}: Es seien \(F\) und \(G\) Formeln, für die gilt \(1+tiefe(F) \le |F| \) und \(1+tiefe(G) \le |G| \).\\
\underline{Induktionsschritt}: TODO\\
\underline{Resümee}: Nach dem Prinzip der struktruellen Induktion ergibt sich
damit, dass die Tiefe einer Formel echt kleiner ist als ihre Länge.


\aufgabe{5.3}
\underline{Behauptung}: Die Länge jeder Formel ist nach oben durch \(2^{tiefe(F)+2}-3\) beschränkt:
\[\forall F \in L_{AL} : |F| \le 2^{tiefe(F)+2}-3\]
\underline{Induktionsanfang}: TODO\\
\underline{Induktionsannahme}: Es seien \(F\) und \(G\) Formeln, für die gilt \(2^{tiefe(F)+2}-3\) und \(2^{tiefe(G)+2}-3\).\\
\underline{Induktionsschritt}: TODO\\
\underline{Resümee}: Nach dem Prinzip der struktruellen Induktion ergibt sich
damit, dass die Länge jeder Formel nach oben hin durch \(2^{tiefe(F)+2}-3\) beschränkt ist.


\aufgabe{5.4}


\end{enumerate}
\end{document}

\documentclass[a4paper]{scrartcl}
\usepackage[ngerman]{babel}
\usepackage[utf8]{inputenc}
\usepackage[T1]{fontenc}
\usepackage{lmodern}
\usepackage{amssymb}
\usepackage{amsmath}
\usepackage{enumerate}
\usepackage{scrpage2}\pagestyle{scrheadings}
\usepackage{tikz}
\usetikzlibrary{arrows,automata}

\newcommand{\titleinfo}{Hausaufgaben zum 15.05.2012}
\newcommand{\aufgabe}[1]{\item[\textbf{#1}]}

\title{\titleinfo}
\author{Arne Feil}
\date{\today}
\ihead{FGI1}
\chead{Arne Feil, Sven-Hendrik Haase, Christian Darsow-Fromm}
\ohead{\today}
\setheadsepline{1pt}
\newcommand{\qed}{\quad \square}
\newcommand{\terminal}[1]{\langle #1 \rangle }
\everymath{\displaystyle}
\begin{document}
%\maketitle
\begin{enumerate}

\aufgabe{6.3}
\begin{enumerate}[1.]
 \item

  \textbf{allgemeingültig, erfüllbar}

  $$((A\vee B))\vee (A\Rightarrow A))$$
  Da die Teilformel $(A\Rightarrow A)$ offensichtlich allgemeingültig und mit \textit{oder} zum anderen Teil verknüpft ist, ist die gesamte Formel allgemeingültig.

  $$((A\vee B)\vee\neg B)$$
  $B$ oder $\neg B$ muss immer wahr sein.

  $$(A\vee B)\Rightarrow(A\vee\neg A))$$
  $(A\vee\neg A)$ ist immer wahr, daher auch die ganze Formel.

  \textbf{kontingent, erfüllbar, falsifizierbar}

  $$((A\vee B)\wedge(A\Rightarrow B))$$
  Mit z.B. $A=0$ und $B=0$ ist die Formel unerfüllt und mit $A=1$ und $B=1$ ist sie erfüllt.

  $$((A\vee B)\Rightarrow(\neg A\wedge \neg A))$$
  Mit $A=1$ und $B=1$ ist sie unerfüllt und mit $A=0$ und $B=0$ erfüllt.

  $$((A\vee B)\wedge(B\Leftrightarrow A))$$
  Mit $A=0$ und $B=1$ ist sie unerfüllt und mit $A=1$ und $B=0$ erfüllt.

  \textbf{unerfüllbar, falsifizierbar}
  $$((A\vee B)\wedge(A\Leftrightarrow B)\wedge\neg A)$$
  %TODO Begründung

  $$((A\vee B)\Leftrightarrow(\neg A\wedge\neg B))$$
  %TODO Begründung

  $$((A\vee B)\wedge(\neg B\Leftrightarrow A)\wedge(A\Leftrightarrow B))$$
  %TODO Begründung

  \item
\end{enumerate}

\aufgabe{6.4}

\aufgabe{6.5}


\end{enumerate}
\end{document}

\documentclass[a4paper,10pt]{scrartcl}
\usepackage[ngerman]{babel}
\usepackage[utf8]{inputenc}
\usepackage[T1]{fontenc}
\usepackage{lmodern}
\usepackage{amssymb}
\usepackage{amsmath}
\usepackage{enumerate}
\usepackage{scrpage2}\pagestyle{scrheadings}
\usepackage{tikz}
\usetikzlibrary{arrows,automata}

\newcommand{\titleinfo}{Hausaufgaben zum 22.05.2012}
\newcommand{\aufgabe}[1]{\item[\textbf{#1}]}

\title{\titleinfo}
\author{Arne Feil}
\date{\today}
\ihead{FGI1}
\chead{Arne Feil, Sven-Hendrik Haase, Christian Darsow-Fromm}
\ohead{\today}
\setheadsepline{1pt}
\newcommand{\qed}{\quad \square}
\newcommand{\terminal}[1]{\langle #1 \rangle }
\everymath{\displaystyle}
\begin{document}
%\maketitle
\begin{enumerate}
\aufgabe{7.3}

\aufgabe{7.4}

\begin{enumerate}[1.]
 \item

 \item
  \begin{enumerate}[a)]
   \item
    $R_a=\frac{B\Rightarrow A}{A}$ \\
    Nach Satz 6.7 ist die Inferenzrelation erfüllt, wenn $(B\Rightarrow A)\vDash A$ gilt, also $A$ immer erfüllt ist, wenn auch $(B\Rightarrow A)$ wahr ist.\\
    Dies ist jedoch nicht der Fall, wenn $B=A=0$ ist. Dann ist $(B\Rightarrow A)=1$, nicht jedoch $A$. Daher gilt diese Inferenzrelation nicht.
  \end{enumerate}
  % Diese Aufgabe habe ich fertig, bin nur noch nicht ganz zum Aufschreiben gekommen.
\end{enumerate}



\end{enumerate}
\end{document}

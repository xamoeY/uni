\documentclass[a4paper]{scrartcl}
\usepackage[ngerman]{babel}
\usepackage[utf8]{inputenc}
\usepackage[T1]{fontenc}
\usepackage{lmodern}
\usepackage{amssymb}
\usepackage{amsmath}
\usepackage{enumerate}
\usepackage{scrpage2}\pagestyle{scrheadings}
\usepackage{tikz}
\usetikzlibrary{arrows,automata}

\newcommand{\titleinfo}{Hausaufgaben zum 24.04.2012}
\newcommand{\aufgabe}[1]{\item[\textbf{#1}]}

\title{\titleinfo}
\author{Arne Feil}
\date{\today}
\ihead{FGI1}
\chead{Arne Feil, Sven-Hendrik Haase, Christian Darsow-Fromm}
\ohead{\today}
\setheadsepline{1pt}
\newcommand{\qed}{\quad \square}
\everymath{\displaystyle}
\begin{document}
%\maketitle
\begin{enumerate}

\aufgabe{3.3}
\begin{enumerate}[1.]
 \item
 % Muss hier unbedingt eine \epsilon-Verbindung enthalten sein?
 \begin{tikzpicture}[->, auto, node distance=2.2cm, >=latex]
	\tikzstyle{every initial by arrow}=[initial text=,->, >=stealth]

    \node[initial, state] (A) {$q_1$};
    \node[state] (B) [right of=A] {$q_2$};
    \node[state] (C) [right of=B] {$q_3$};
    \node[accepting, state] (D) [right of=C] {$q_4$};

    \draw (A) edge [bend left] node {0} (B);
    \draw (B) edge [bend left] node {1} (C);
    \draw (C) edge [bend left] node {0} (D);
 \end{tikzpicture}

 \item
 \begin{tikzpicture}[->, auto, node distance=2.2cm, >=latex]
	\tikzstyle{every initial by arrow}=[initial text=,->, >=stealth]

    \node[initial, state] (A) {$q_1$};
    \node[state] (B) [right of=A] {$q_2$};
    \node[state] (C) [right of=B] {$q_3$};
    \node[accepting, state] (D) [right of=C] {$q_4$};

    \draw (A) edge [bend left] node {$\epsilon$} (B);
    \draw (B) edge [loop above] node {0} (B);
    \draw (B) edge [bend left] node {$\epsilon$} (C);
    \draw (C) edge [loop above] node {0,1} (C);
    \draw (C) edge [bend left] node {0} (D);
 \end{tikzpicture}

 \item
 \begin{tikzpicture}[->, auto, node distance=2.2cm, >=latex]
	\tikzstyle{every initial by arrow}=[initial text=,->, >=stealth]

    \node[initial, state] (A) {$q_1$};
    \node[state] (B) [right of=A] {$q_2$};
    \node[accepting, state] (C) [right of=B] {$q_3$};
    \node[state] (D) [right of=C] {$q_4$};

    \draw (A) edge [bend left] node {$\epsilon$} (B);
    \draw (B) edge [loop above] node {0} (B);
    \draw (B) edge [bend left] node {$\epsilon$} (C);
    \draw (C) edge [bend left] node {1} (D);
    \draw (D) edge [loop above] node {0,1} (D);
    \draw (D) edge [bend left] node {0} (C);
 \end{tikzpicture}

\end{enumerate}


\aufgabe{3.4}
$$R^0_{1,1}=\{\epsilon,a\},\ R^0_{1,2}=\{a\},\ R^0_{2,1}=\{\emptyset\},\ R^0_{2,2}=\{\epsilon, b\}$$
$$R^0_{2,3}=\{a\},\ R^0_{3,2}=\{b\},\ R^0_{3,3}=\{\epsilon\}$$
$$R^0_{1,3}=\{\emptyset\},\ R^0_{3,1}=\{b\}$$
\begin{align}
 R^1_{1,3} &= R^0_{1,2} R^0_{2,3} = \{a\}\{a\}\\
 R^2_{1,3} &= R^0_{1,1}R^1_{1,3} \cup R^0_{1,2}R^0_{2,2}R^0_{2,3} = \{a\}^+\{a\}\cup \{a\}\{b\}^*\{a\}\\
 R^3_{1,3} &= R^2_{1,3} \cup R^0_{1,1} R^0_{1,2}R^0_{2,2}R^0_{2,3} \cup R^1_{1,3} (R^0_{3,1} R^2_{1,3})^*\\
		   &= \{a\}^+\{b\}^*\{a\} \left(\{b\}\{a\}^+\{b\}^*\{a\}\right)^*
\end{align}

\end{enumerate}
\end{document}

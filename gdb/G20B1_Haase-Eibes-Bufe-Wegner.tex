\documentclass[12pt]{article}
\usepackage[utf8]{inputenc}
\usepackage{amssymb,amsmath}
\usepackage{hyperref}
\author{Sven-Hendrik Haase, Ingo Eibes, Alexander Bufe, Benjamin Wegner}
\title{GDB HA zum 25.10.12}
\date{\today}
\begin{document}
\setcounter{secnumdepth}{0}
\maketitle

\section{Aufgabe 1}
\subsection{(a)}
Ein Informationssystem besteht aus Menschen und Maschinen. Es gibt eine Kommunikation zwischen beiden.
Menschen benötigen z.B. Informationen von der Maschine oder wollen welche auf dieser Speichern.
Aufgaben eines rechnergestützen Informationssystems sind Erfassung, Speicherung undransformation von Informationen.

\subsection{(b)}
Physische Datenunabhängigkeit bedeutet, dass Änderungen z.B. am physischen Speichersystem keine Auswirkungen auf die Datenbank haben.
Logische Datenunabhängigkeit bedeutet, dass Änderungen am Datenbankschema keine Auswirkung auf die Daten haben.


\subsection{(c)}
Ein Beispiel wäre ein Krankenhaus. Es müssen Termine (Ankunft der Patienten, Behandlungen) und Räume (Bettenbelegung) verwaltet werden, sowie Informationen über die Patienten gespeichert und wieder aufgerufen werden.


(Beispiel 2 fehlt)


\section{Aufgabe 2}
blubb


\section{Aufgabe 3}

Bei einem Stromausfall an Zeitpunkt A kann (muss aber nicht) es sein,
dass die Änderung bereits auf die Platte geschrieben wurde und somit das Geld abgebucht,
aber niemandem hinzugebucht wurde, also einfach weg ist. 
Bei einem Stromausfall an Zeitpunkt B passiert, wenn die Daten bereits auf die Platte geschrieben wurden nichts.
Wurden sie das nicht, kann es wieder sein, dass das Geld zwar abgebucht, aber nicht gutgeschrieben wurde.
\end{align}
\end{document}

\documentclass[a4paper,10pt,]{scrartcl}
\usepackage[ngerman]{babel}
\usepackage[utf8]{inputenc}
\usepackage[T1]{fontenc}
\usepackage{lmodern}
\usepackage{amssymb}
\usepackage{amsmath}
\usepackage{booktabs}
\usepackage{enumerate}
\usepackage{scrpage2}\pagestyle{scrheadings}

% \usepackage{sans}

\usepackage{tikz}
\usepackage{tikz-er2}
% \usepackage{geometry}
% \usepackage{lscape}
\usepackage{graphicx}
\usepackage{setspace}

  
\usepackage{pdfpages}

\ihead{GDB G20\_B3}
\chead{}
\ohead{Elena Noll, Sven-Hendrik Haase}
\pagestyle{scrheadings}
\setheadsepline{1pt}
\newcommand{\qed}{\quad \square}

\begin{document}
\section*{Aufgabe 1}
\subsection*{a)}
\begin{verbatim}
CREATE TABLE Abteilungen (
    aid     int NOT NULL PRIMARY KEY,
    name    varchar(50) UNIQUE
);

CREATE TABLE Personal (
    pid     int NOT NULL PRIMARY KEY,
    vorname varchar(50) NOT NULL,
    nachname    varchar(50) NOT NULL,
    geburt  date    NULL,
    wohnort varchar(50) NULL,
    abteilung   int NOT NULL,
    CONSTRAINT abteilung_id FOREIGN KEY(abteilung) REFERENCES Abteilungen(aid)
);

CREATE TABLE Projekte (
    pr_id   int NOT NULL PRIMARY KEY,
    name    varchar(100) NOT NULL,
    leiter  int NOT NULL,
    budget  int NOT NULL CHECK(0 < budget < 200000),
    CONSTRAINT personal_id1 FOREIGN KEY(leiter) REFERENCES Personal(pid)
);

CREATE TABLE ProjektArbeiter (
    prid    int NOT NULL,
    pid     int NOT NULL,
    PRIMARY KEY(prid, pid),
    CONSTRAINT projekte_id2 FOREIGN KEY(prid) REFERENCES Projekte(pr_id),
    CONSTRAINT personal_id2 FOREIGN KEY(pid) REFERENCES Personal(pid)
);

\end{verbatim}

\newpage
\subsection*{b)}
Wenn die Bedingung sofort geprüft wird, wird bei einer Verletzung dieser
bereits nach der Ausführung des SQL-Statements ein Fehler geworfen. Bei
verzögterem Check wird erst beim commit der Transkation der Fehler geworfen.
Letzteres könnte zur Folge haben, dass die anderen, unabhängigen
SQL-Statements zwischen den commits ebenfalls nicht ausgeführt werden.\\
Wenn Abteilungen noch einen Fremdschlüssel auf Personal enthalten soll,
bestünde bei der Erstellung beider Tabellen eine zyklische Abhängigkeit zwischen beiden.
Es muss Abteilung vorläufig ohne Fremdschlüssel auf Personal erstellt werden
und dann nachträglich, nachdem Personal erstellt wurde, per ALTER TABLE erst
eine neue Spalte in Abteilung hinzugefügt werden und dann mit ALTER
TABLE ... ADD CONSTRAINT ... REFERENCES die Referenz auf Personal hergestellt
werden.

\subsection*{c)}
\begin{verbatim}
INSERT INTO Abteilungen (aid, name) VALUES (1, "Controlling"),
                                           (4, "Marketing"),
                                           (2, "Einkauf");

INSERT INTO Personal (pid, vorname, nachname, geburt, wohnort, abteilung) VALUES
            (4, "Peter", "Müller", "1962-07-25", "Hamburg", 2),
            (8, "Bianca", "Lohnse", "1982-01-13", "Kiel", 4),
            (11, "Murat", "Sahir", "1990-03-16", "Hamburg", 2),
            (21, "Frank", "Siebenstein", "1975-12-02", "Norderstedt", 1),
            (22, "Bernd", "Schmidt", "1973-11-26", "Norderstedt", 1),
            (24, "Ulrike", "Müller", "1963-10-07", "Hamburg", 2),
            (31, "Jochen", "Fuhrmann", "1958-05-09", "Stade", 2);

INSERT INTO Projekte (prid, name, leiter, budget) VALUES
            (15, "Prozessoptimierung", 22, 10000),
            (36, "B.L.I.C.K.F.A.N.G", 8, 7500);

INSERT INTO ProjektArbeiter (prid, pid) VALUES (36, 8),
                                               (15, 21),
                                               (36, 11),
                                               (15, 22),
                                               (36, 31);
\end{verbatim}

\newpage
\subsection*{d)}
\begin{verbatim}
DROP TABLE ProjektArbeiter;
DROP TABLE Projekte;
DROP TABLE Personal;
DROP TABLE Abteilungen;

DELETE FROM Personal WHERE vorname = "Peter" AND nachname = "Müller";
\end{verbatim}


\end{document}

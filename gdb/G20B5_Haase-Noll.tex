\documentclass[a4paper,12pt,]{scrartcl}
\usepackage[ngerman]{babel}
\usepackage[utf8]{inputenc}
\usepackage[T1]{fontenc}
\usepackage{lmodern}
\usepackage{amssymb}
\usepackage{amsmath}
\usepackage{booktabs}
\usepackage{enumerate}
\usepackage{scrpage2}\pagestyle{scrheadings}

% \usepackage{sans}

\usepackage{tikz}
% \usepackage{geometry}
% \usepackage{lscape}
\usepackage{graphicx}
\usepackage{setspace}

  
\usepackage{pdfpages}

\ihead{GDB G20\_B5}
\chead{}
\ohead{Elena Noll, Sven-Hendrik Haase}
\pagestyle{scrheadings}
\setheadsepline{1pt}
\newcommand{\qed}{\quad \square}

\begin{document}
\section*{Aufgabe 1}
\subsection*{a)}

\subsubsection*{i)}
\begin{verbatim}
CREATE VIEW LetzteVerhandlung AS 
SELECT RID AS Richter,
(SELECT VID FROM Verhandlung WHERE RID = Vorsitz ORDER BY VID DESC LIMIT 1)
AS letzte_verhandlung FROM Richter GROUP BY RID;
\end{verbatim}

\subsubsection*{ii)}
\begin{verbatim}
CREATE VIEW Minderjaehrige AS 
SELECT Nachname, Alter
FROM Person
WHERE Alter < 18;
\end{verbatim}

\subsubsection*{iii)}
\begin{verbatim}
CREATE VIEW Prozessinformationen AS 
SELECT *
FROM Prozess pr, Person p, Straftat s
WHERE pr.Angeklagter = p.PID
AND pr.Anklage = s.SID;
\end{verbatim}

\subsection*{b)}
\subsubsection*{i)}
Geht. Alle Bedingungen erfüllt.

\subsubsection*{ii)}
Geht. Sicherheitsstufe-Bedingung wird verletzt, aber es wird nicht kontrolliert.

\subsubsection*{iii)}
Geht nicht. Sicherheitsstufe-Bedingung wird verletzt und es wird kontrolliert.

\subsubsection*{iv)}
Geht nicht. Datum-Bedingung wird verletzt und es wird kontrolliert.

\subsubsection*{v)}
Geht. Saal-Bedingung wird verletzt, aber es wird nicht kontrolliert.

\subsubsection*{vi)}
Geht. Sicherheitsstufe-Bedingung wird verletzt, aber es wird nicht kontrolliert.

\section*{Aufgabe 2}
\subsection*{a)}
Ein Schema, welches bezüglich der referentiellen Aktionen sicher ist,
stellt sich, dass die referentielle Integrität bei Änderungsoperationen
nicht gefährdet wird.

\subsection*{b)}
Es kann sein, dass eine Person gelöscht wird, die aber Leiter der Videothek ist. Hiergegen ist das Schema momentan nicht geschützt.

\section*{Aufgabe 3}
\subsection*{S1}
\subsection*{a)}
A = 115 \\
B = 5
\subsection*{b)}
A wird von \(T_2\) gelesen und geschrieben, aber \(T_1\) überschreibt A, ohne den neuen Wert von \(T_2\) zu respektieren.
\subsection*{c)}
Nicht serialisierbar, denn Änderungen gehen verloren.

\subsection*{S2}
\subsection*{a)}
A = 300 \\
B = 5
\subsection*{b)}
keine Abhängigkeiten
\subsection*{c)}
Seriell, denn \(T_1\) wird komplett vor \(T_2\) ausgeführt.

\subsection*{S3}
\subsection*{a)}
A = 300 \\
B = 5
\subsection*{b)}
\(T_1\) vor \(T_2\). 
\subsection*{c)}
Serialisierbar, denn die Änderungen der Transaktionen werden untereinander berücksichtigt.

\subsection*{S4}
\subsection*{a)}
A = 195 \\
B = 5
\subsection*{b)}
\(T_1\) schreibt A nachdem \(T_2\) A gelesen hat. \(T_2\) schreibt A danach, ohne die Änderungen von \(T_1\) zu berücksichtigen.
\subsection*{c)}
Nicht serialisierbar, denn Änderungen gehen verloren.

\subsection*{S5}
\subsection*{a)}
A = 305 \\
B = 195
\subsection*{b)}
\(T_2\) vor \(T_1\). 
\subsection*{c)}
Seriell, denn \(T_2\) wird vollkommen vor \(T_1\) ausgeführt. 

\subsection*{S6}
\subsection*{a)}
A = 195 \\
B = 5
\subsection*{b)}
B wird von \(T_1\) neu geschrieben, aber zuvor wird es von \(T_2\) gelesen, sodass die Änderungen nicht beachtet überschrieben werden.
\subsection*{c)}
Nicht serialisierbar, denn \(T_2\) überschreibt die Änderungen von \(T_1\) vollständig.


\end{document}